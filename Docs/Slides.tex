\documentclass{beamer}

\usepackage[utf8]{inputenc}

\newcommand{\codett}[1]{\textsf{\footnotesize{}#1}}
\newcommand{\earley}[4]{#1\overset{#4}{\mapsto}#2\cdot#3}
\newcommand{\St}[1]{\mathbb{S}_{#1}}

\begin{document}

	\title{Formally Verified Earley Parsing in Agda}
	\author{Elias Forsberg}
	\date{\today}
	
	\begin{frame}
		\maketitle
		\centering
		\begin{tabular}{c}
			Supervisor: Andreas Abel \\ Examiner: Nils-Anders Danielsson
		\end{tabular}\\
		\flushleft
		{\tiny Department of Computer Science and Engineering\\}
		{\tiny \sc Chalmers University of Technology\\}
		{\tiny \sc University of Gothenburg\\}
	
	\end{frame}
	
	% This title can be a bit much to unpack. Let's start from the back and
	% work our way to the front

	\begin{frame}
		\frametitle{What's Agda?}

		\begin{itemize}
			\item Functional programming language
			\item Haskell-like \emph{(ish)} syntax
			\item Dependently typed
			\item Dependent types allow expressing constructing proofs as terms 
				in the language
		\end{itemize}
	\end{frame}

	\begin{frame}
		\frametitle{What's Agda?}
		% Code example (++-assoc from report?) as quick intro to agda
	\end{frame}

	\begin{frame}
		\frametitle{What's Parsing?}
		% Short intro to parsing, what is a cfg, what do are ambiguous things

		\begin{itemize*}
			\item Analyze the structure of a sequence of symbols
			\item Structures are called grammars
			\item We're interested in context-free grammars
		\end{itemize*}\\
		\vspace{0.5cm}
		\centering
		\begin{tabular}{c}
			$X \mapsto \alpha$ 
		\end{tabular}
		\vspace{2cm}

	\end{frame}

	\begin{frame}
		\frametitle{What's Parsing?}
		% Short intro to parsing, what is a cfg, what do are ambiguous things

		\begin{itemize*}
			\item Analyze the structure of a sequence of symbols
			\item Structures are called grammars
			\item We're interested in context-free grammars
		\end{itemize*}\\
		\vspace{0.5cm}
		\centering
		\begin{tabular}{c}
			$Z \mapsto Ya$ 
			$Y \mapsto b$ 
		\end{tabular}
		\vspace{2cm}
	\end{frame}

	% probably good idea to discuss _⊢_∈_ and friends here

	\begin{frame}
		\frametitle{Earley's Algorithm}
		% Walk through parsing a string using Earley's Algorithm. All 
		% explanation done during parsing
	\end{frame}

	\begin{frame}
		\frametitle{Earley's Algorithm}
		\centering
		\begin{tabular}{rcl}
			$E$ & $ \mapsto $ & $V$ \\
			$E$ & $ \mapsto $ & $E$ $O$ $E$ \\
			$O$ & $ \mapsto $ & `$+$' \\
			$O$ & $ \mapsto $ & `$-$' \\
			$V$ & $ \mapsto $ & `$x$' \\
			$V$ & $ \mapsto $ & `$y$' \\
			$V$ & $ \mapsto $ & `$z$'
		\end{tabular}
		% This table, or parts of it should be part of several (all?) of the 
		% slides where the earley states are shown.
	\end{frame}

	\begin{frame}
		\frametitle{Earley's Algorithm}
		\centering
		\begin{tabular}{|c|c|c|c|}
			\hline
			$\St{0}$ & $\St{1}$ & $\St{2}$ & $\St{3}$ \\
			\hline
			$\earley{S_0}{\varepsilon}{E}{0}$ & & & \\
			\hline
		\end{tabular}
	\end{frame}

	\begin{frame}
		\frametitle{Earley's Algorithm}
		\centering
		\begin{tabular}{|c|c|c|c|}
			\hline
			$\St{0}$ & $\St{1}$ & $\St{2}$ & $\St{3}$ \\
			\hline
			$\earley{S_0}{\varepsilon}{E}{0}$ & & & \\
			$\earley{E}{\varepsilon}{V}{0}$   & & & \\
			$\earley{E}{\varepsilon}{EOE}{0}$ & & & \\
			\hline
		\end{tabular}
	\end{frame}

	\begin{frame}
		\frametitle{Earley's Algorithm}
		\centering
		\begin{tabular}{|c|c|c|c|}
			\hline
			$\St{0}$                          & $\St{1}$ & $\St{2}$ & $\St{3}$ \\
			\hline
			$\earley{S_0}{\varepsilon}{E}{0}$ & $\earley{V}{x}{\varepsilon}{0}$ & &\\
			$\earley{E}{\varepsilon}{V}{0}$   & & & \\
			$\earley{E}{\varepsilon}{EOE}{0}$ & & & \\
			$\earley{V}{\varepsilon}{x}{0}$   & & & \\
			$\earley{V}{\varepsilon}{y}{0}$   & & & \\
			$\earley{V}{\varepsilon}{z}{0}$   & & & \\
			\hline
		\end{tabular}
	\end{frame}

	\begin{frame}
		\frametitle{Earley's Algorithm}
		\centering
		\begin{tabular}{|c|c|c|c|}
			\hline
			$\St{0}$                          & $\St{1}$ & $\St{2}$ & $\St{3}$ \\
			\hline
			$\earley{S_0}{\varepsilon}{E}{0}$ & $\earley{V}{x}{\varepsilon}{0}$   &  & \\
			$\earley{E}{\varepsilon}{V}{0}$   & $\earley{E}{V}{\varepsilon}{0}$   &  & \\
			$\earley{E}{\varepsilon}{EOE}{0}$ & $\earley{S_0}{E}{\varepsilon}{0}$ &  & \\
			$\earley{V}{\varepsilon}{x}{0}$   & $\earley{E}{E}{OE}{0}$            &  & \\
			$\earley{V}{\varepsilon}{y}{0}$   &                                   &  & \\
			$\earley{V}{\varepsilon}{z}{0}$   &                                   &  & \\
			\hline
		\end{tabular}
	\end{frame}

	\begin{frame}
		\frametitle{Earley's Algorithm}
		\centering
		\begin{tabular}{|c|c|c|c|}
			\hline
			$\St{0}$                          & $\St{1}$                          &\St{2} & $\St{3}$ \\
			\hline
			$\earley{S_0}{\varepsilon}{E}{0}$ & $\earley{V}{x}{\varepsilon}{0}$   & $\earley{O}{+}{\varepsilon}{1}$ & \\
			$\earley{E}{\varepsilon}{V}{0}$   & $\earley{E}{V}{\varepsilon}{0}$   & & \\
			$\earley{E}{\varepsilon}{EOE}{0}$ & $\earley{S_0}{E}{\varepsilon}{0}$ & & \\
			$\earley{V}{\varepsilon}{x}{0}$   & $\earley{E}{E}{OE}{0}$            & & \\
			$\earley{V}{\varepsilon}{y}{0}$   & $\earley{O}{\varepsilon}{+}{0}$   & & \\
			$\earley{V}{\varepsilon}{z}{0}$   & $\earley{O}{\varepsilon}{-}{0}$   & & \\
			\hline
		\end{tabular}
	\end{frame}

	\begin{frame}
		\frametitle{Earley's Algorithm}
		\centering
		\begin{tabular}{|c|c|c|c|}
			\hline
			$\St{0}$                          & $\St{1}$                          & $\St{2}$                         $ & $\St{3}$ \\
			\hline
			$\earley{S_0}{\varepsilon}{E}{0}$ & $\earley{V}{x}{\varepsilon}{0}$   & $\earley{O}{+}{\varepsilon}{1}  $ & $\earley{V}{y}{\varepsilon}{2}  $ \\
			$\earley{E}{\varepsilon}{V}{0}$   & $\earley{E}{V}{\varepsilon}{0}$   & $\earley{E}{EO}{E}{0}           $ & $\earley{E}{V}{\varepsilon}{2}  $ \\
			$\earley{E}{\varepsilon}{EOE}{0}$ & $\earley{S_0}{E}{\varepsilon}{0}$ & $\earley{E}{\varepsilon}{V}{2}  $ & $\earley{E}{EOE}{\varepsilon}{0}$ \\
			$\earley{V}{\varepsilon}{x}{0}$   & $\earley{E}{E}{OE}{0}$            & $\earley{E}{\varepsilon}{EOE}{2}$ & $\earley{E}{E}{OE}{2}           $ \\
			$\earley{V}{\varepsilon}{y}{0}$   & $\earley{O}{\varepsilon}{+}{1}$   & $\earley{V}{\varepsilon}{x}{2}  $ & $\earley{S_0}{E}{\varepsilon}{0}$ \\
			$\earley{V}{\varepsilon}{z}{0}$   & $\earley{O}{\varepsilon}{-}{1}$   & $\earley{V}{\varepsilon}{y}{2}  $ & $\earley{O}{\varepsilon}{+}{3}  $ \\
			$                                 &                                   & $\earley{V}{\varepsilon}{z}{2}  $ & $\earley{O}{\varepsilon}{-}{3}  $ \\
			\hline
		\end{tabular}
	\end{frame}
	
	\begin{frame}
		\frametitle{Earley's Algorithm - Impletentation / Formalization}
		% I implemented the stuff we just talked about.
		% Talk about nullable rules and the solution I used?
		% No need to show any code -- just outline what was done
	\end{frame}

	\begin{frame}
		\frametitle{Soundness proof}
	\end{frame}

	\begin{frame}
		\frametitle{Completeness proof}
	\end{frame}

	\begin{frame}
		\frametitle{Trying it out}
		% Show the parser parsing some input. Maybe the
		% "equal number of a's and b's" grammar.
	\end{frame}

	\begin{frame}
		\frametitle{Conclusion}
		% Managed to show soundness and completenes for the parser.
		% Parser is slow, probably due to poor choice of data structures.
	\end{frame}
	
\end{document}
