\documentclass{article}

\usepackage[utf8]{inputenc}

\usepackage{natbib}
\usepackage{xcolor}
\usepackage{hyperref}
\hypersetup{
	hidelinks = true,
}

\usepackage{graphicx}

\usepackage[colorinlistoftodos]{todonotes}

\usepackage{parskip}
\setlength{\parskip}{10pt} 

\usepackage{tikz}
\usetikzlibrary{arrows, decorations.markings}

\usepackage{chngcntr}
\counterwithout{figure}{section}

\begin{document}

\begin{titlepage}
  

\centering
  
  
{\scshape\LARGE Master thesis project proposal\\}
  
\vspace{0.5cm}
  
{\huge\bfseries Formally Verified Parser Generator\\}
  
\vspace{2cm}
  
{\Large Elias Forsberg (eliasfo@student.chalmers.se)\\}
  
\vspace{1.0cm}
  
{\large Suggested Supervisor at CSE: Andreas Abel\\}
  
\vspace{1.5cm}
  
{\large Relevant completed courses:\par}
  
{\itshape %List (course code, name of course)
	\begin{tabular}{cl}
		DAT151 & Programming language technology \\
		DAT350 & Types for programs and proofs \\
		TDA342 & Advanced functional programming \\
		TDA283 & Compiler construction \\
	\end{tabular} \\
}
\vfill

\vfill
  
{\large \today\\} 

\end{titlepage}

\section{Introduction}

	% Briefly describe and motivate the project, and convince the reader of the
	% importance of the proposed thesis work.  A good introduction will answer
	% these questions: Why is addressing these challenges significant for
	% gaining new knowledge in the studied domain? How and where can this new
	% knowledge be applied?

	Agda is a functional programming language with a strong type system that
	supports dependent types. These allow expressing strong invariants and
	properties of data structures and programs, and even constructive proofs
	using the type system. These proofs enable programmers to formally verify
	the correctness of programs and libraries. While the language itself has
	been shown to be sound, neither its parser, nor later stages of its
	compiler have been formally verified. Errors in the implementations of
	these might change the semantics of compiled programs and proofs,
	undermining their usefulness.

	Parser generators are commonly used for parsers of programming languages,
	and several different generators exist today, such as Happy~\cite{Happy},
	Bison~\cite{Bison}, and Menhir~\cite{Menhir}. Parser generators are used
	for several reasons: BNF descriptions of grammars are usually more concise
	compared to other representations, such as hand written parsers. Generated
	parsers that are specialized for a language can also be faster compared to
	algorithms that parse any grammar and have to read and analyze the grammar
	of the target language during each parse. Parsers are often required to be
	fast, as they are the first step in what is usually many different passes
	of the entire compilation process. To this end, many parser generators
	generate parsers based on the LR(1) or LALR(1) parsing algorithms. These
	algorithms are designed mainly to be fast, but are unfortunately not able
	to parse all context free grammars.  Other algorithms, such as the Earley
	parsing algorithm~\cite{Earley}, are able to parse all CFG's, while
	maintaining an O(n) time complexity for most unambiguous grammars.

	The BNF Converter~\cite{BNFC} is a parser and lexer generator written in
	Haskell, which uses a labeled BNF description of the target language to
	generate an abstract syntax tree representation, a parser and a pretty
	printer for the target language. BNFC Supports multiple different back-end
	languages (that is, target languages for the generated parser itself), but
	none of these, nor other parts of BNFC have been formally verified. As with
	compilers, errors in parser generators can change the semantics of
	programs, yielding incorrect results. 
	
	% A formally A fverified parser genrator could be used to generate parsers
	% for several A fdifferent formally verified compilers
	
	% Nice if BNFC was guarantied to generate correct parsers.

	Verified compilers have received significant attention in the research
	community. CompCert~\cite{Leroy} is a formally verified compiler for a
	subset of the C programming language. It is written, and proved to preserve
	the semantics of compiled programs, in Coq, a proof assistant similar to
	Agda. This compiler does not, however, do any form of verification of the
	parsing step, but starts from the AST level. If verified parsers could be
	easily generated for use in different programming languages, it would be
	possible to have compilers that are fully verified from source code in its
	original text format to the output of the compiler.
	
	The aim of this thesis is to create a parser generator that can create a
	parser for any context-free grammar, using (possibly among others) the
	Earley parsing algorithm, and to formally verify, using Agda, that all
	generated parsers are correct\footnote{Sound and complete}. 
	
	% This parser generator could then potentially be used to generate a parser
	% for Agda.

% \section{Problem} 

	% This section is optional. It may be used if there is a need to describe
	% the problem that you want to solve in more technical detail and if this
	% problem description is too extensive to fit in the introduction.

\section{Context}

	% Use one or two relevant and high quality references for providing
	% evidence from the literature that the proposed study indeed includes
	% scientific and engineering challenges, or is related to existing ones.
	% Convince the reader that the problem addressed in this thesis has not
	% been solved prior to this project.

	Jourdan and Pottier~\cite{Jourdan} demonstrated a method for verifying the
	correctness of generated LR parsers. They used the fact that LR parsers are
	stack automata guided by a finite-state automaton, and designed a verifier
	that can check whether a given automaton conforms to a specified grammar.
	As such, they did not prove the correctness of the parser generator itself,
	and their method can not verify parsers capable of parsing any context-free
	grammar.
	
	Firsov and Uustalu~\cite{Firsov14}, verified a CYK parsing algorithm using
	Agda. The CYK parsing algorithm only works for grammars in Chomsky normal
	form, which the grammars for most target languages will not be in. This
	means that the grammar will first have to be normalized, a process which
	also should be verified in order to have a fully verified parsing process.
	Fortunately, Firsov and Uustalu later verified a normalization algorithm
	for context free grammars.~\cite{Firsov15} The normalization process can,
	however, increase the size of the grammar, up to a quadratic increase. This
	might be undesirable for some situations where very large grammars are
	used.

	While these works are closely related, this thesis will aim to verify a
	parser generator for context-free grammars. This means that a parser will
	be created for each target grammar, and ideally that this parser could be
	emitted in several different target languages. The thesis will also use the
	Earley parsing algorithm, which has not previously been formally verified.

\section{Goals and Challenges}

	% Describe your contribution with respect to concepts, theory and technical
	% goals. Ensure that the scientific and engineering challenges stand out so
	% that the reader can easily recognize that you are planning to solve an
	% advanced problem. 
	
	The goals for this thesis will be to use Agda as a programming language to:
	
	\begin{enumerate}

%		\item \label{lbnf}
%
%			Parse LBNF grammar files. In the first iteration, this can be done
%			by binding a BNFC-generated Haskell parser for LBNF to Agda.
%			Finally, this can be done by boot-strapping.

		\item \label{lex}
			
			Implement a lexer, and prove its correctness.

		\item \label{par}
			
			Implement the Earley parser, and prove its correctness.
		
		\item \label{dsl}
			
			Design a simple intermediate representation for generated lexers
			and parsers. 

		\item \label{lexgen}
			
			Implement a lexer generator, and prove correctness of its output
			for all grammars.

		\item \label{pargen}
			
			Implement a parser generator, and prove correctness of its output
			for all grammars.
		
		\item \label{backend}

			Implement a compiler from the intermediate representation to some
			back-end language, such as Agda or C, and prove correctness of the
			output of that compiler, for all inputs.

	\end{enumerate}
	
	It is hard to estimate how long steps \ref{lexgen}-\ref{backend} will take.
	Concluding at step \ref{par} would be sufficient for completing the
	project, if correctness proofs of the implemented parser are complete.
	Step \ref{pargen} has room for many optimizations and user experience
	improvements, but any of these would be optional. While step \ref{backend}
	would be necessary for a end-to-end verified compiler, it might be a very
	large task, depending on the target language (or indeed languages, since it
	would be desirable for a parser generator to be able to generate parsers in
	many different languages). This project will therefore most likely conclude
	before step \ref{backend}.

	One additional goal of this project is to parse and use BNFC-compatible
	Labeled BNF syntax files to use as definitions of grammars. The parser
	generator should also use these to generate an AST representation for the
	grammar in the target language.

\section{Approach}

	% Various scientific approaches are appropriate for different challenges
	% and project goals. Outline and justify the ones that you have selected.
	% For example, when your project considers systematic data collection, you
	% need to explain how you will analyze the data, in order to address your
	% challenges and project goals.

	% One scientific approach is to use formal models and rigorous mathematical
	% argumentation to address aspects like correctness and efficiency. If this
	% is relevant, describe the related algorithmic subjects, and how you plan
	% to address the studied problem. For example, if your plan is to study the
	% problem from a computability aspect, address the relevant issues, such as
	% algorithm and data structure design, complexity analysis, etc.  If you
	% plan to develop and evaluate a prototype, briefly describe your plans to
	% design, implement, and evaluate your prototype by reviewing at most two
	% relevant issues, such as key functionalities and their evaluation
	% criteria.

	% The design and implementation should specify prototype properties, such
	% as functionalities and performance goals, e.g., scalability, memory,
	% energy.  Motivate key design selection, with respect to state of the art
	% and existing platforms, libraries, etc.  

	% When discussing evaluation criteria, describe the testing environment,
	% e.g., test-bed experiments, simulation, and user studies, which you plan
	% to use when assessing your prototype. Specify key tools, and preliminary
	% test-case scenarios. Explain how and why you plan to use the evaluation
	% criteria in order to demonstrate the functionalities and design goals.
	% Explain how you plan to compare your prototype to the state of the art
	% using the proposed test-case evaluation scenarios and benchmarks. 

	Creating the initial parser and lexer (steps \ref{lex} and \ref{par} above)
	will be useful for parsing LBNF syntax files in order to read the
	definitions of the parsers that should be generated. These initial proofs
	of a parser and lexer are also expected to be simpler than the full
	generator proofs, and might severe as guiding steps for the full proofs.

	The intermediate representation for parsers and lexers (step \ref{dsl}
	above) will ideally be designed with simple semantics. The semantics of
	this language could be defined using an interpreter written in Agda. This
	will help disconnect the correctness proof of the parser generator from the
	semantics of any potential target languages, as these may be quite complex.
	We hope that this will lighten the proof burden for the parser and lexer
	generators.

	% Lexer generator and parser generator proofs the real meat of this
	% project.

	The final step of the parser generator, compiling the intermediate
	representation to the target parser language (step \ref{backend} above),
	while necessary for a complete end-to-end verified parser, is potentially a
	lot of work to verify. The semantics of each target language would first
	have to be formalized, if they are not already, and it would then be
	necessary to prove that the compiler preserves the semantics of the
	intermediate representation under this formalization. An \emph{un}verified
	compiler for the intermediate representation might be constructed for this
	project, so that the parser generator can be used even if all steps have
	not been verified.

	% Proofs themselves can't be tested/evaluated?  Compare speed to LR and
	% LALR parsers?

% \section{References}

	% Reference all sources that are cited in your proposal using, e.g. the
	% APA, Harvard2, or IEEE3 style.

\newpage

\bibliographystyle{plain}

\bibliography{lit}

\end{document}
