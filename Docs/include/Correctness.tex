
\chapter{Correctness proofs}
	
	Here, we discuss the structure of the correctness proofs for the 
	implementation discussed in chapter \ref{Earleys}. First, we discuss 
	soundness of the generated item sets, and continue from there to their 
	completeness. Finally backtracking from the item sets to create parse 
	trees, and ambiguous parses, which are of particular interest for the 
	Earley parsing algorithm, are discussed.

	\section{Soundness}

		The soundness proof for our implementation is straightforward 
		throughout. First, we provide a definition of sound items sets: 

		\begin{code}

			Valid : ∀ {w v} -> Item w v -> Set
			Valid {w} {v} i = G ⊢ Item.u i / v ⟶* Item.Y i / Item.β i

			Sound : ∀ {w v} -> WSet w v -> Set
			Sound (start rs) = ∀ {i} -> i ∈ rs -> Valid i
			Sound (step ω rs) = Sound ω × (∀ {i} -> i ∈ rs -> Valid i)

			H : ∀ {v w} {ω : WSet w v} -> Sound ω -> (∀ {i} -> i ∈ Sₙ ω -> Valid i)
			H {ω = start rs} s = s
			H {ω = step ω rs} s = snd s

		\end{code}
		
		An item is \codett{Valid} if it can be derived using the Earley rules,
		and a set of item sets is \codett{Sound} if all items in it are valid.
		This definition of soundness is both simple and intuitive. Proving that
		the implementation always preserves these properties is also very
		straightforward, as the algorithm never modifies items that are added
		to an item set, and the implementation generates new items in three
		different ways, matching the \emph{scan}, \emph{predict}, and
		\emph{complete} steps in the algorithm also present in our Earley
		propositions.

		First, we will take a look at the scanning step. We will consider 
		\codett{scanr₀} to be sound if all its generated items are valid, given
		valid input. This allows us to show that \codett{scanr} preserves 
		soundness for \codett{WSet}s:

		\begin{code}

			sound-scanr₀ : ∀ {a v w} -> ∀ rs ->
			  (∀ {i} -> i ∈ rs -> Valid i) ->
			  (∀ {i} -> i ∈ scanr₀ {w} {v} a rs -> Valid i)
			sound-scanr₀ ε f ()
			sound-scanr₀ ((X ∘ u ↦ α ∘ ε) ∷ rs) f p = sound-scanr₀ rs (f ∘ in-tail) p
			sound-scanr₀ ((X ∘ u ↦ α ∘ l Y ∷ β) ∷ rs) f p = sound-scanr₀ rs (f ∘ in-tail) p
			sound-scanr₀ {a} ((X ∘ u ↦ α ∘ r b ∷ β) ∷ rs) f p with decidₜ a b
			... | no x = sound-scanr₀ rs (f ∘ in-tail) p
			... | yes refl with p
			...            | in-head    = scanner (f in-head)
			...            | in-tail p₁ = sound-scanr₀ rs (f ∘ in-tail) p₁

			sound-scanr : ∀ {a w v} -> (ω : WSet w (a ∷ v)) ->
			  Sound ω -> Sound (scanr a ω)
			sound-scanr (start rs) s = s , sound-scanr₀ rs s
			sound-scanr (step w rs) (s , s₁) = s , s₁ , sound-scanr₀ rs s₁

		\end{code}

		The structure of the proof matches the structure of the original 
		definitions closely, due to the way normalizaitons are axiomatic to the
		type system. If there is no input, there is no output, so nothing 
		needs to be shown. Even when there is, \codett{scanr₀} only generates
		new items from items that expect a terminal, and only does so when that 
		terminal equals the next token in the input sequence. Finally, the 
		proof is structured so that an arbitrary generated item is chosen, and 
		that item should then be shown to be valid. The argument \codett{p} 
		specifies for which of the generated items the proof is desired, and if 
		the one generated at this recursion level is not the correct one, we 
		must continue the proof search.

		Next, we continue in a similar fashion for the complete step:

		\begin{code}

			sound-compl₀ : ∀ {u v w} (w : WSet w v) ->
			  Sound w -> (∀ {i} -> i ∈ compl₀ {u} {v} w -> Valid i)
			sound-compl₀ {u} {v} w s p           with eq-T* u v
			sound-compl₀ {v} {v} w s p           | yes refl = H s p
			sound-compl₀ {u} {v} (start rs) s () | no x
			sound-compl₀ {u} {v} (step w rs) s p | no x = sound-compl₀ w (fst s) p
		
		\end{code}

		As \codett{sound-compl₀} is nothing more than a lookup function to find
		the correct item set to use from a \codett{WSet}, its soundness is 
		trivial when the soundness of the original item set is given. 
		\codett{sound-compl₁} is more interesting:

		\begin{code}

			sound-compl₁ : ∀ {u v w} ->
			  (i : Item w v) ->
			  (p : Item.β i ≡ ε) ->
			  (q : Item.u i ≡ u) ->
			  (rs : Item w u *) ->
			  Valid i -> (∀ {j} -> j ∈ rs -> Valid j) ->
			  (∀ {j} -> j ∈ compl₁ i p rs -> Valid j)

			sound-compl₁ i refl refl ε v f ()

			sound-compl₁ i refl refl ((Y ∘ u₁ ↦ α₁ ∘ ε) ∷ rs) v f q = 
			  sound-compl₁ i refl refl rs v (f ∘ in-tail) q

			sound-compl₁ i refl refl ((Y ∘ u₁ ↦ α₁ ∘ r a ∷ β) ∷ rs) v f q = 
			  sound-compl₁ i refl refl rs v (f ∘ in-tail) q

			sound-compl₁ i refl refl ((Y ∘ u₁ ↦ α₁ ∘ l Z ∷ β) ∷ rs) v f q with decidₙ X Z
			... | no x = sound-compl₁ i refl refl rs v (f ∘ in-tail) q
			... | yes refl with q = complet (f in-head) v
			...            | in-head     = complet (f in-head) v
			...            | (in-tail q) = sound-compl₁ i refl refl rs v (f ∘ in-tail) q

		\end{code}
		
		Here, we show, given a valid item that can be completed ($\beta =
		\epsilon$), and an item set from which matching completion items can be 
		found, that \codett{compl₁} only generates valid items if all input 
		items are valid. Like with the scanning step, we only need to focus on 
		the cases where items are actually generated, and when they are, if the 
		current one is the one for which a proof is sought. The generated items
		also match the \emph{complete} step of our Earley proposition 
		perfectly, so the proof for the relevant item is also very simple.

		The soundness proof for \codett{sound-compl₂} is then trivial:
		
		\begin{code}

			sound-compl₂ : ∀ {v w} ->
			  (i : Item w v) ->
			  (p : Item.β i ≡ ε) ->
			  Valid i -> (ω : WSet w v) -> Sound ω ->
			  (∀ {j} -> j ∈ compl₂ i p ω -> Valid j)
			sound-compl₂ i p v ω s q =
			  sound-compl₁ i p refl (compl₀ ω) v (sound-compl₀ ω s) q

		\end{code}
		
		As \codett{predict₁} works in very much the same way as
		\codett{compl₂}, both their type proof structure are very similar (the
		only difference being that \codett{compl₂} requires $\beta$ to start
		with a non-terminal instead of being empty), and so the soundness proof
		for \codett{predict₁} will not be discussed in further detail.
		
		Soundness of the merged step \codett{pred-comp} of the \codett{predict}
		and \codett{compl} steps is also completely trivial, as there are no
		modification to existing items nor generation of new ones except for
		what is done in \codett{compl₂} and \codett{predict₁}. Their type
		signatures are shown below:

		\begin{code}

			sound-pred-comp₀ : ∀ {v w} ->
			  (i : Item w v) -> 
			  (w : WSet w v) ->
			  Valid i -> Sound w ->
			  (∀ {j} -> j ∈ Σ.proj₁ (pred-comp₀ i w) -> Valid j)

			sound-pred-comp₁ : ∀ {w v} (ω : WSet w v) -> ∀ ss rs ->
			  (∀ {i} -> i ∈ ss -> Valid i) ->
			  (∀ {i} -> i ∈ rs -> Valid i) ->
			  Sound ω -> (∀ {i} -> i ∈ pred-comp₁ ω ss rs -> Valid i)
		
		\end{code}

		\codett{sound-pred-comp₂} is a bit more complex:

		\begin{code}

			sound-pred-comp₂ : ∀ {w v} (ω : WSet w v) -> ∀ ss rs m p q ->
			  (∀ {i} -> i ∈ ss -> Valid i) ->
			  (∀ {i} -> i ∈ rs -> Valid i) ->
			  Sound ω -> Sound (pred-comp₂ ω ss rs m p q)
			sound-pred-comp₂ ω ss rs zero () q f g s
			sound-pred-comp₂ ω ss ε (suc m) p q f g s = soundₙ ω f s
			sound-pred-comp₂ ω ss rs@(r₁ ∷ _) (suc m) p q f g s =
			  let x₁ = pred-comp₁ ω ss rs in
			  let x₂ = x₁ \\ (rs ++ ss) in
			  let p₁ = wf-pcw₃ (Σ.proj₀ all-rules) p q  in
			  let p₂ = wf-pcw₂ x₁ (rs ++ ss) q in
			  let p₃ = sound-pred-comp₁ ω ss rs f g s in
			  let p₄ = s-pcw₀ Valid p₃ in
			  sound-pred-comp₂ ω (rs ++ ss) x₂ m p₁ p₂ (in-either Valid g f) p₄ s
		
		\end{code}

		\begin{code}

			sound-pred-comp : ∀ {w v} {ω : WSet w v} -> 
			  Sound ω -> Sound (pred-comp ω)

			sound-step₀ : ∀ {w a v} {ω : WSet w (a ∷ v)} -> 
			  Sound ω -> Sound (step₀ ω)

			sound-parse₀ : ∀ {w v} -> {ω : WSet w v} -> 
			  Sound ω -> Sound (parse₀ ω)

			sound-parse : ∀ w -> Sound (parse w)

		\end{code}

	\section{Completeness}

%		\begin{code}
%
%			_≋_ : ∀ {t u v X α β} -> (i : Item t v) -> G ∙ t ⊢ u / v ⟶* X / α ∙ β -> Set
%			_≋_ {t} {u} {v} {X} {α} {β} i g = (Item.Y i , Item.u i , Item.α i , Item.β i) ≡ (X , u , α , β)
%
%			eq-prop : {a b : T *} (P : Item a b -> Set) (i : Item b b) -> a ≡ b -> Set
%			eq-prop P i refl = P i
%
%			Complete₀ : ∀ {t v} -> WSet t v -> Set
%			Complete₀ {t} {v} ω = ∀ {u Y α β} ->
%			  (i : Item t v) ->
%			  (g : G ∙ t ⊢ u / v ⟶* Y / α ∙ β) ->
%			  i ≋ g ->
%			  i ∈ Sₙ ω
%
%			mutual
%			  Complete : ∀ {v w} -> WSet w v -> Set
%			  Complete ω = Complete₀ ω × Complete* ω
%
%			  Complete* : ∀ {v w} -> WSet w v -> Set
%			  Complete* (start rs) = ⊤
%			  Complete* (step ω rs) = Complete ω
%
%			complete-scanner₀ : ∀ {t u v X α β} -> ∀ a rs ->
%			  (i : Item t (a ∷ v)) ->
%			  (j : Item t v) ->
%			  (g : G ∙ t ⊢ u / a ∷ v ⟶* X / α ∙ r a ∷ β) ->
%			  i ≋ g -> i ∈ rs ->
%			  j ≋ scanner g ->
%			    j ∈ scanner₀ a rs
%			complete-scanner₀ a ε i j g refl () refl
%			complete-scanner₀ a ((X ∘ u ↦ α ∘ ε) ∷ rs)       i j g refl (in-tail p) refl = complete-scanner₀ a rs i j g refl p refl
%			complete-scanner₀ a ((X ∘ u ↦ α ∘ l Y ∷ β) ∷ rs) i j g refl (in-tail p) refl = complete-scanner₀ a rs i j g refl p refl
%			complete-scanner₀ a ((X ∘ u ↦ α ∘ r b ∷ β) ∷ rs) i j g refl p           refl with decidₜ a b
%			complete-scanner₀ a ((X ∘ u ↦ α ∘ r b ∷ β) ∷ rs) i j g refl in-head     refl | yes refl = in-head
%			complete-scanner₀ a ((X ∘ u ↦ α ∘ r b ∷ β) ∷ rs) i j g refl (in-tail p) refl | yes refl = in-tail (complete-scanner₀ a rs i j g refl p refl)
%			complete-scanner₀ a ((X ∘ u ↦ α ∘ r b ∷ β) ∷ rs) i j g refl in-head     refl | no x = void (x refl)
%			complete-scanner₀ a ((X ∘ u ↦ α ∘ r b ∷ β) ∷ rs) i j g refl (in-tail p) refl | no x = complete-scanner₀ a rs i j g refl p refl
%
%			complete-scanner : ∀ {t u v X α β} -> ∀ a ->
%			  (ω : WSet t (a ∷ v)) ->
%			  Complete ω ->
%			  (i : Item t (a ∷ v)) ->
%			  (j : Item t v) ->
%			  (g : G ∙ t ⊢ u / a ∷ v ⟶* X / α ∙ r a ∷ β) ->
%			  i ≋ g ->
%			  j ≋ scanner g ->
%			    j ∈ Sₙ (scanner a ω)
%			complete-scanner a ω@(start rs) c  i j g refl refl with (fst c) i g refl
%			complete-scanner a ω@(start rs) c  i j g refl refl | d = complete-scanner₀ a (Sₙ ω) i j g refl d refl
%			complete-scanner a ω@(step _ rs) c i j g refl refl with (fst c) i g refl
%			complete-scanner a ω@(step _ rs) c i j g refl refl | d = complete-scanner₀ a (Sₙ ω) i j g refl d refl
%
%			test : ∀ {a t v} ->
%			  {P : Item t v -> Set} ->
%			  (Σ λ u -> u ++ (a ∷ v) ≡ t) ∣ v ≡ t ->
%
%			  (∀ {u X α β} ->
%			    (g : G ∙ t ⊢ u / a ∷ v ⟶* X / α ∙ r a ∷ β) ->
%			    (i : Item t v) -> (i ≋ scanner g) ->
%			    P i
%			  ) ->
%
%			  (∀ {β} ->
%			    (z : t ≡ v) ->
%			    (x : (CFG.start G , β) ∈ CFG.rules G) ->
%			    (i : Item v v) ->
%			    i ≋ initial x -> eq-prop P i z
%			  ) ->
%
%			  (∀ {u X Y α β δ} (x : (Y , δ) ∈ CFG.rules G) ->
%			    (i j : Item t v) ->
%			    (g : G ∙ t ⊢ u / v ⟶* X / α ∙ l Y ∷ β) ->
%			    (i ≋ g) -> P i ->
%			    (j ≋ predict x g) -> P j
%			  ) ->
%
%			  (∀ {u w X Y α β γ} ->
%			    (j k : Item t v) ->
%			    (g : G ∙ t ⊢ u / w ⟶* X / α ∙ l Y ∷ β) ->
%			    (h : G ∙ t ⊢ w / v ⟶* Y / γ ∙ ε) ->
%			    (j ≋ h) -> P j ->
%			    (k ≋ complet g h) -> P k
%			  ) ->
%
%			  (∀ {u X α β} ->
%			    (g : G ∙ t ⊢ u / v ⟶* X / α ∙ β) ->
%			    (i : Item t v) ->
%			    i ≋ g ->
%			    P i
%			  )
%			test s f ini h c (initial x) i refl = ini refl x i refl
%			test {a} s f ini h c (scanner g) i p = case test₀ s g of λ {refl -> f g i p}
%			test s f ini h c (predict x g) i@(X ∘ u ↦ α ∘ β [ χ ∘ ψ ]) refl =
%			  h x (_ ∘ _ ↦ _ ∘ _ [ in-g g ∘ suff-g₁ g ]) i g refl (test s f ini h c g _ refl) refl
%			test s f ini h c (complet g g₁) i p =
%			  c (_ ∘ _ ↦ _ ∘ _ [ in-g g₁ ∘ suff-g₂ g ]) i g g₁ refl (test s f ini h c g₁ _ refl) p
%
%			complete-predict₀ : ∀ {t u v X Y α β δ} ->
%			  (ψ₀ : Σ λ s -> s ++ v ≡ t) ->
%			  (i j : Item t v) ->
%			  (g : G ∙ t ⊢ u / v ⟶* X / α ∙ l Y ∷ β) ->
%			  (h : G ∙ t ⊢ v / v ⟶* Y / ε ∙ δ) ->
%			  (p : i ≋ g) ->
%			  (q : j ≋ h) ->
%			  (rs : (Σ λ t -> (t ∈ CFG.rules G) × (fst t ≡ Y)) *) ->
%			  (Σ λ e -> σ (Y , δ) e ∈ rs) ->
%			  j ∈ predict₀ ψ₀ i (≋-β i g p) rs
%			complete-predict₀ ψ₀ (X ∘ u ↦ α ∘ l Y ∷ β) (Y ∘ u₁ ↦ ε ∘ δ) g h refl refl ε (σ p₁ ())
%			complete-predict₀ ψ₀ (X ∘ u ↦ α ∘ l Y ∷ β) (Y ∘ u₁ ↦ ε ∘ δ) g h refl refl (σ (Y , γ) (p , refl) ∷ rs) q with eq-α γ δ
%			complete-predict₀ ψ₀ (X ∘ u ↦ α ∘ l Y ∷ β) (Y ∘ u₁ ↦ ε ∘ δ) g h refl refl (σ (Y , δ) (p , refl) ∷ rs) q | yes refl = in-head
%			complete-predict₀ ψ₀ (X ∘ u ↦ α ∘ l Y ∷ β) (Y ∘ u₁ ↦ ε ∘ δ) g h refl refl (σ (Y , δ) (p , refl) ∷ rs) (σ (q , refl) in-head) | no x = void (x refl)
%			complete-predict₀ ψ₀ (X ∘ u ↦ α ∘ l Y ∷ β) (Y ∘ u₁ ↦ ε ∘ δ) g h refl refl (σ (Y , γ) (p , refl) ∷ rs) (σ q₁ (in-tail q₀)) | no x =
%			  in-tail (complete-predict₀ ψ₀ _ _ g h refl refl rs (σ q₁ q₀))
%
%			complete-predict₁ : ∀ {t u v X Y α β δ} ->
%			  (ω : WSet t v) ->
%			  (i j : Item t v) ->
%			  (g : G ∙ t ⊢ u / v ⟶* X / α ∙ l Y ∷ β) ->
%			  (x : CFG.rules G ∋ (Y , δ)) ->
%			  (p : i ≋ g) ->
%			  j ≋ predict x g ->
%			  j ∈ predict₁ i (≋-β i g p) ω
%			complete-predict₁ ω i@(X ∘ u ↦ α ∘ l Y ∷ β) j g x refl q =
%			  complete-predict₀ _ i j g (predict x g) refl q (lookup Y (CFG.rules G)) (lookup-sound x)
%
%			complete-deduplicate : ∀ {w v i} -> (as : Item w v *) -> i ∈ as -> i ∈ Σ.proj₁ (deduplicate as)
%			complete-deduplicate ε ()
%			complete-deduplicate (x ∷ as) p with elem eq-item x (Σ.proj₁ (deduplicate as))
%			complete-deduplicate (x ∷ as) in-head     | yes x₁ = x₁
%			complete-deduplicate (x ∷ as) (in-tail p) | yes x₁ = complete-deduplicate as p
%			complete-deduplicate (x ∷ as) in-head     | no x₁ = in-head
%			complete-deduplicate (x ∷ as) (in-tail p) | no x₁ = in-tail (complete-deduplicate as p)
%
%			complete₁-pred-comp₀ : ∀ {t u v X Y α β δ} ->
%			  (ω : WSet t v) ->
%			  (x : (Y , δ) ∈ CFG.rules G) ->
%			  (i j : Item t v) ->
%			  (g : G ∙ t ⊢ u / v ⟶* X / α ∙ l Y ∷ β) ->
%			  i ≋ g ->
%			  j ≋ predict x g ->
%			    j ∈ Σ.proj₁ (pred-comp₀ i ω)
%			complete₁-pred-comp₀ ω x i@(Y ∘ u ↦ α ∘ l Z ∷ β) j@(Z ∘ u₁ ↦ ε ∘ β₁ [ χ₁ ∘ ψ₁ ]) g refl refl =
%			  let x₁ = complete-predict₁ ω i j g x refl refl in
%			  complete-deduplicate (predict₁ i refl ω) x₁
%
%			complete₂-pred-comp₀ : ∀ {t u v w X Y α β γ} ->
%			  (ω : WSet t w) ->
%			  (i j : Item t w) ->
%			  (g : G ∙ t ⊢ u / v ⟶* X / α ∙ l Y ∷ β) ->
%			  (h : G ∙ t ⊢ v / w ⟶* Y / γ ∙ ε) ->
%			  i ≋ h ->
%			  j ≋ complet g h ->
%			    j ∈ Σ.proj₁ (pred-comp₀ i ω)
%			complete₂-pred-comp₀ ω i j g h refl refl =
%			  complete-deduplicate (complete₂ i refl ω) {!!}
%
%			complete₁-pred-comp₁ : ∀ {t u v X Y α β δ ss} -> ∀ rs ->
%			  (ω : WSet t v)
%			  (x : (Y , δ) ∈ CFG.rules G) ->
%			  (i j : Item t v) ->
%			  (g : G ∙ t ⊢ u / v ⟶* X / α ∙ l Y ∷ β) ->
%			  i ≋ g ->
%			  i ∈ rs ->
%			  j ≋ predict x g ->
%			    j ∈ pred-comp₁ ω ss rs
%			complete₁-pred-comp₁ {ss = ss} (i  ∷ rs) ω x i j g o in-head q =
%			  in-l (complete₁-pred-comp₀ (Wₙ ω ss) x i j g o q)
%			complete₁-pred-comp₁ (r₁ ∷ rs) ω x i j g o (in-tail p) q =
%			  in-r (complete₁-pred-comp₁ rs ω x i j g o p q)
%
%			complete₁₀-pred-comp₂ : ∀ {t v ss rs m p q} ->
%			  (ω : WSet t v) ->
%			  (i : Item t v) ->
%			  i ∈ ss ->
%			  i ∈ Sₙ (pred-comp₂ ω ss rs m p q)
%			complete₁₀-pred-comp₂ {ss = ss} {rs} {zero} {()} ω i s
%			complete₁₀-pred-comp₂ {ss = ss} {ε} {suc m} {p} (start rs) i s = s
%			complete₁₀-pred-comp₂ {ss = ss} {ε} {suc m} {p} (step ω rs) i s = s
%			complete₁₀-pred-comp₂ {ss = ss} {rs@(_ ∷ _)} {suc m} {p} {q} ω i s =
%			  complete₁₀-pred-comp₂ {m = m} ω i (in-r s)
%
%			complete₁₁-pred-comp₂ : ∀ {t v ss rs m p q} ->
%			  (ω : WSet t v) ->
%			  (i : Item t v) ->
%			  i ∈ rs ->
%			  i ∈ Sₙ (pred-comp₂ ω ss rs m p q)
%			complete₁₁-pred-comp₂ {ss = ss} {rs} {zero} {()} ω i s
%			complete₁₁-pred-comp₂ {ss = ss} {ε} {suc m} {p} ω i ()
%			complete₁₁-pred-comp₂ {ss = ss} {rs@(_ ∷ _)} {suc m} {p} ω i s =
%			  complete₁₀-pred-comp₂ {m = m} ω i (in-l s)
%
%			complete₁₂-pred-comp₂ : ∀ {t v ss rs m p q} ->
%			  (ω : WSet t v) ->
%			  ∀ {u X Y α β δ} ->
%			  (x : (Y , δ) ∈ CFG.rules G) ->
%			  (i j : Item t v) ->
%			  (g : G ∙ t ⊢ u / v ⟶* X / α ∙ l Y ∷ β) ->
%			  i ≋ g -> i ∈ rs ->
%			  j ≋ predict x g -> j ∈ Sₙ (pred-comp₂ ω ss rs m p q)
%			complete₁₂-pred-comp₂ {ss = ss} {rs} {zero} {p = ()} ω x i j g p p' q
%			complete₁₂-pred-comp₂ {ss = ss} {ε} {suc m} ω x i j g p () q
%			complete₁₂-pred-comp₂ {ss = ss} {rs@(_ ∷ _)} {suc m} ω x i j g p p' q =
%			  let x₁ = pred-comp₁ ω ss rs in
%			  let x₂ = x₁ \\ (rs ++ ss) in
%			  let y₁ = complete₁-pred-comp₁ rs ω x i j g p p' q in
%			  let y₂ = include-\\ {as = x₁} {bs = rs ++ ss} y₁ in
%			  case in-lr x₂ (rs ++ ss) y₂ of
%			    λ { (r z) → complete₁₀-pred-comp₂ {m = m} ω j z
%			      ; (l z) → complete₁₁-pred-comp₂ {m = m} ω j z
%			      }
%
%			Nx : ∀ {t v} -> Item t v * -> Item t v * -> Set
%			Nx {t} {v} ss rs =
%			  ∀ {u X Y α β δ} ->
%			  (x : (Y , δ) ∈ CFG.rules G) ->
%			  (i j : Item t v) ->
%			  (g : G ∙ t ⊢ u / v ⟶* X / α ∙ l Y ∷ β) ->
%			  i ≋ g -> i ∈ ss ->
%			  j ≋ predict x g ->
%			    j ∈ (rs ++ ss)
%
%			nx' : ∀ {t v rs ss} (ω : WSet t v) ->
%			  Nx ss rs ->
%			  Nx (rs ++ ss) (pred-comp₁ ω ss rs \\ (rs ++ ss))
%			nx' {rs = ε} {ss} ω nx x i j g z₁ z₂ z₃ = nx x i j g z₁ z₂ z₃
%			nx' {rs = rs@(r₁ ∷ rs₀)} {ss} ω nx x i j g z₁ z₂ z₃ =
%			  case in-lr rs ss z₂ of
%			    λ { (r x₁) →
%			      let x₁ = nx x i j g z₁ x₁ z₃  in
%			      in-r x₁
%			    ; (l x₁) →
%			      let y₁ = complete₁-pred-comp₁ rs ω x i j g z₁ x₁ z₃ in
%			      include-\\ y₁
%			    }
%
%			complete₁-pred-comp₂ : ∀ {t v} -> ∀ ss rs m p q ->
%			  (ω : WSet t v) ->
%			  Nx ss rs ->
%			  ∀ {u X Y α β δ}
%			  (x : (Y , δ) ∈ CFG.rules G) ->
%			  (i j : Item t v) ->
%			  (g : G ∙ t ⊢ u / v ⟶* X / α ∙ l Y ∷ β) ->
%			  i ≋ g ->
%			  i ∈ Sₙ (pred-comp₂ ω ss rs m p q) ->
%			  j ≋ predict x g ->
%			    j ∈ Sₙ (pred-comp₂ ω ss rs m p q)
%			complete₁-pred-comp₂ ss rs            zero    () q ω           nx x i j g z₁ z₂ z₃
%			complete₁-pred-comp₂ ss ε             (suc m) p  q (start rs)  nx x i j g z₁ z₂ z₃ =
%			  nx x i j g z₁ z₂ z₃
%			complete₁-pred-comp₂ ss ε             (suc m) p  q (step ω rs) nx x i j g z₁ z₂ z₃ =
%			  nx x i j g z₁ z₂ z₃
%			complete₁-pred-comp₂ ss rs@(r₁ ∷ rs₀) (suc m) p  q ω           nx x i j g z₁ z₂ z₃ =
%			  let p₁ = wf-pcw₃ (Σ.proj₀ all-rules) p q in
%			  let q₁ = wf-pcw₂ (pred-comp₁ ω ss rs) (rs ++ ss) q in
%			  complete₁-pred-comp₂ (rs ++ ss) _ m p₁ q₁ ω (nx' ω nx) x i j g z₁ z₂ z₃
%
%			Nx₂ : ∀ {t v} -> Item t v * -> Item t v * -> Set
%			Nx₂ {t} {v} ss rs =
%			  ∀ {u w X Y α β γ} ->
%			  (j k : Item t v) ->
%			  (g : G ∙ t ⊢ u / w ⟶* X / α ∙ l Y ∷ β) ->
%			  (h : G ∙ t ⊢ w / v ⟶* Y / γ ∙ ε) ->
%			  j ≋ h -> j ∈ ss ->
%			  k ≋ complet g h ->
%			    k ∈ (rs ++ ss)
%
%			nx₂' : ∀ {t v rs ss} (ω : WSet t v) ->
%			  Complete* ω ->
%			  Nx₂ ss rs ->
%			  Nx₂ (rs ++ ss) (pred-comp₁ ω ss rs \\ (rs ++ ss))
%			nx₂' {rs = ε} {ss} ω c nx x i j g z₁ z₂ z₃ = nx x i j g z₁ z₂ z₃
%			nx₂' {rs = rs@(r₁ ∷ rs₀)} {ss} ω c nx x i j g z₁ z₂ z₃ =
%			  case in-lr rs ss z₂ of
%			    λ { (r x₁) →
%			      let x₁ = nx x i j g z₁ x₁ z₃  in
%			      in-r x₁
%			    ; (l x₁) → {!!}
%			    }
%
%			complete₂-pred-comp₂ : ∀ {t v} -> ∀ ss rs m p q ->
%			  (ω : WSet t v) ->
%			  Complete* ω ->
%			  Nx₂ ss rs ->
%			  ∀ {u w X Y α β γ}
%			  (j k : Item t v) ->
%			  (g : G ∙ t ⊢ u / w ⟶* X / α ∙ l Y ∷ β) ->
%			  (h : G ∙ t ⊢ w / v ⟶* Y / γ ∙ ε) ->
%			  j ≋ h -> j ∈ Sₙ (pred-comp₂ ω ss rs m p q) ->
%			  k ≋ complet g h ->
%			    k ∈ Sₙ (pred-comp₂ ω ss rs m p q)
%			complete₂-pred-comp₂ ss rs            zero    () q ω           c nx x i j g z₁ z₂ z₃
%			complete₂-pred-comp₂ ss ε             (suc m) p  q (start rs)  c nx x i j g z₁ z₂ z₃ =
%			  nx x i j g z₁ z₂ z₃
%			complete₂-pred-comp₂ ss ε             (suc m) p  q (step ω rs) c nx x i j g z₁ z₂ z₃ =
%			  nx x i j g z₁ z₂ z₃
%			complete₂-pred-comp₂ ss rs@(r₁ ∷ rs₀) (suc m) p  q ω           c nx x i j g z₁ z₂ z₃ =
%			  let p₁ = wf-pcw₃ (Σ.proj₀ all-rules) p q in
%			  let q₁ = wf-pcw₂ (pred-comp₁ ω ss rs) (rs ++ ss) q in
%			  complete₂-pred-comp₂ (rs ++ ss) _ m p₁ q₁ ω c (nx₂' ω c nx) x i j g z₁ z₂ z₃
%
%			complete-pred-comp₂ : ∀ {a t v ss rs m p q} ->
%			  (Σ λ u -> u ++ (a ∷ v) ≡ t) ∣ v ≡ t ->
%			  (ω : WSet t v) ->
%			  Complete* ω ->
%			  Nx ss rs ->
%			  Nx₂ ss rs ->
%			  (∀ {u X α β} ->
%			    (g : G ∙ t ⊢ u / a ∷ v ⟶* X / α ∙ r a ∷ β) ->
%			    (i : Item t v) -> (i ≋ scanner g) ->
%			    i ∈ Sₙ (pred-comp₂ ω ss rs m p q)
%			  ) ->
%			  (∀ {β} ->
%			    (z : t ≡ v) ->
%			    (x : (CFG.start G , β) ∈ CFG.rules G) ->
%			    (i : Item v v) ->
%			    i ≋ initial x -> eq-prop (λ i -> i ∈ Sₙ (pred-comp₂ ω ss rs m p q)) i z
%			  ) ->
%			  ∀ {u X α β} ->
%			  (g : G ∙ t ⊢ u / v ⟶* X / α ∙ β) ->
%			  (i : Item t v) ->
%			  i ≋ g -> i ∈ Sₙ (pred-comp₂ ω ss rs m p q)
%			complete-pred-comp₂ {a} {t} {v} {ss} {rs} {m} {p} {q} h ω c nx nx₂ s u g i e =
%			  test {P = λ i -> i ∈ Sₙ (pred-comp₂ ω ss rs m p q)}
%			    h
%			    s
%			    u
%			    (complete₁-pred-comp₂ ss rs m p q ω nx)
%			    (complete₂-pred-comp₂ ss rs m p q ω c nx₂)
%			    g i e
%
%			complete₀-pred-comp : ∀ {a t v} ->
%			  (Σ λ u -> u ++ (a ∷ v) ≡ t) ∣ v ≡ t ->
%			  (ω : WSet t v) ->
%			  Complete* ω ->
%			  Nx ε (Σ.proj₁ (deduplicate (Sₙ ω))) ->
%			  Nx₂ ε (Σ.proj₁ (deduplicate (Sₙ ω))) ->
%			  (∀ {u X α β} ->
%			    (g : G ∙ t ⊢ u / a ∷ v ⟶* X / α ∙ r a ∷ β) ->
%			    (i : Item t v) -> (i ≋ scanner g) ->
%			    i ∈ (Σ.proj₁ (deduplicate (Sₙ ω)))
%			  ) ->
%			  (∀ {β} ->
%			    (z : t ≡ v) ->
%			    (x : (CFG.start G , β) ∈ CFG.rules G) ->
%			    (i : Item v v) ->
%			    i ≋ initial x -> eq-prop (_∈ Σ.proj₁ (deduplicate (Sₙ ω))) i z
%			  ) ->
%			  ∀ {u X α β} ->
%			  (g : G ∙ t ⊢ u / v ⟶* X / α ∙ β) ->
%			  (i : Item t v) ->
%			  i ≋ g -> i ∈ Sₙ (pred-comp ω)
%			complete₀-pred-comp {a} {t} s ω c nx nx₂ fx fx₂ g i p =
%			  let x₁ = deduplicate (Sₙ ω) in
%			  let x₂ = (unique-++ (Σ.proj₁ x₁) ε (Σ.proj₀ x₁) u-ε λ ()) in
%			  complete-pred-comp₂ {p = ≤ₛ (≤-self _)} {q = x₂} s ω c nx nx₂
%			    (λ g₁ i₁ x → complete₁₁-pred-comp₂ {p = ≤ₛ (≤-self _)} {q = x₂} ω i₁ (fx g₁ i₁ x))
%			    (λ {refl x i₁ x₃ → complete₁₁-pred-comp₂ {p = ≤ₛ (≤-self _)} {q = x₂} ω i₁ (fx₂ refl x i₁ x₃)})
%			    g i p
%
%			complete₃-pred-comp₂ : ∀ {t v} -> ∀ ss rs m p q ->
%			  (ω : WSet t v) ->
%			  Complete* ω ->
%			  Complete* (pred-comp₂ ω ss rs m p q)
%			complete₃-pred-comp₂ ss rs zero () q ω c
%			complete₃-pred-comp₂ ss ε (suc m) p q (start rs) c = top
%			complete₃-pred-comp₂ ss ε (suc m) p q (step ω rs) c = c
%			complete₃-pred-comp₂ ss rs@(_ ∷ _) (suc m) p q ω c =
%			  let x₁ = pred-comp₁ ω ss rs in
%			  let x₂ = x₁ \\ (rs ++ ss) in
%			  let p₁ = wf-pcw₃ (Σ.proj₀ all-rules) p q in
%			  let q₁ = wf-pcw₂ x₁ (rs ++ ss) q in
%			  complete₃-pred-comp₂ (rs ++ ss) x₂ m p₁ q₁ ω c
%
%			complete₁-pred-comp : ∀ {t v} ->
%			  (ω : WSet t v) ->
%			  Complete* ω ->
%			  Complete* (pred-comp ω)
%			complete₁-pred-comp ω =
%			  let x₁ = deduplicate (Sₙ ω) in
%			  let x₂ = (unique-++ (Σ.proj₁ x₁) ε (Σ.proj₀ x₁) u-ε λ ()) in
%			  complete₃-pred-comp₂ ε _ _ (≤ₛ (≤-self _)) x₂ ω
%
%			complete₂-pred-comp : ∀ {t v} ->
%			  (ω : WSet t v) ->
%			  Complete* ω ->
%			  (∀ {u X α β} ->
%			    (g : G ∙ t ⊢ u / v ⟶* X / α ∙ β) ->
%			    (i : Item t v) ->
%			    i ≋ g -> i ∈ Sₙ (pred-comp ω)
%			  ) ->
%			  Complete (pred-comp ω)
%			complete₂-pred-comp ω c f = (λ i g x → f g i x) , complete₁-pred-comp ω c
%
%			complete-pred-comp : ∀ {a t v} ->
%			  (Σ λ u -> u ++ (a ∷ v) ≡ t) ∣ v ≡ t ->
%			  (ω : WSet t v) ->
%			  Complete* ω ->
%			  (∀ {u X α β} ->
%			    (g : G ∙ t ⊢ u / a ∷ v ⟶* X / α ∙ r a ∷ β) ->
%			    (i : Item t v) -> (i ≋ scanner g) ->
%			    i ∈ Sₙ ω
%			  ) ->
%			  (∀ {β} ->
%			    (z : t ≡ v) ->
%			    (x : (CFG.start G , β) ∈ CFG.rules G) ->
%			    (i : Item v v) ->
%			    i ≋ initial x -> eq-prop (_∈ Sₙ ω) i z
%			  ) ->
%			  Complete (pred-comp ω)
%			complete-pred-comp {a} {t} s ω c fx fx₂ =
%			  let
%			    x₅ = complete₀-pred-comp s ω c (λ x i j g x₁ ()) (λ j k g h x ())
%			      (λ {g i x → complete-deduplicate (Sₙ ω) (fx g i x)})
%			      λ {refl x i x₁ → complete-deduplicate (Sₙ ω) (fx₂ refl x i x₁)}
%			  in
%			  complete₂-pred-comp ω c x₅
%
%			complete-step : ∀ {a₀ a t v} ->
%			  (Σ λ u -> u ++ (a₀ ∷ a ∷ v) ≡ t) ∣ (a ∷ v) ≡ t ->
%			  (ω : WSet t (a ∷ v)) ->
%			  Complete* ω ->
%			  (∀ {u X α β} ->
%			    (g : G ∙ t ⊢ u / a₀ ∷ a ∷ v ⟶* X / α ∙ r a₀ ∷ β) ->
%			    (i : Item t (a ∷ v)) -> (i ≋ scanner g) ->
%			    i ∈ Sₙ ω
%			  ) ->
%			  (∀ {β} ->
%			    (z : t ≡ a ∷ v) ->
%			    (x : (CFG.start G , β) ∈ CFG.rules G) ->
%			    (i : Item (a ∷ v) (a ∷ v)) ->
%			    i ≋ initial x -> eq-prop (_∈ Sₙ ω) i z
%			  ) ->
%			  ∀ {u X α β} ->
%			  (i : Item t (a ∷ v)) ->
%			  (j : Item t v) ->
%			  (g : G ∙ t ⊢ u / a ∷ v ⟶* X / α ∙ r a ∷ β) ->
%			  i ≋ g ->
%			  j ≋ scanner g ->
%			    j ∈ Sₙ (step ω)
%			complete-step {a₀} {a} s ω c fx fx₂ i j g refl refl =
%			  let
%			    x₁ = complete-pred-comp s ω c fx fx₂
%			  in complete-scanner a (pred-comp ω) x₁ i j g refl refl
%
%			complete*-step : ∀ {a₀ a t v} ->
%			  (Σ λ u -> u ++ (a₀ ∷ a ∷ v) ≡ t) ∣ (a ∷ v) ≡ t ->
%			  (ω : WSet t (a ∷ v)) ->
%			  Complete* ω ->
%			  (∀ {u X α β} ->
%			    (g : G ∙ t ⊢ u / a₀ ∷ a ∷ v ⟶* X / α ∙ r a₀ ∷ β) ->
%			    (i : Item t (a ∷ v)) -> (i ≋ scanner g) ->
%			    i ∈ Sₙ ω
%			  ) ->
%			  (∀ {β} ->
%			    (z : t ≡ a ∷ v) ->
%			    (x : (CFG.start G , β) ∈ CFG.rules G) ->
%			    (i : Item (a ∷ v) (a ∷ v)) ->
%			    i ≋ initial x -> eq-prop (_∈ Sₙ ω) i z
%			  ) ->
%			  Complete* (step ω)
%			complete*-step s ω c fx fx₂ =
%			  complete-pred-comp s ω c fx fx₂
%
%			complete-parse₀ : ∀ {a t v} ->
%			  (Σ λ u -> u ++ (a ∷ v) ≡ t) ∣ v ≡ t ->
%			  (ω : WSet t v) ->
%			  Complete* ω ->
%			  (∀ {u X α β} ->
%			    (g : G ∙ t ⊢ u / a ∷ v ⟶* X / α ∙ r a ∷ β) ->
%			    (i : Item t v) -> (i ≋ scanner g) ->
%			    i ∈ Sₙ ω
%			  ) ->
%			  (∀ {β} ->
%			    (z : t ≡ v) ->
%			    (x : (CFG.start G , β) ∈ CFG.rules G) ->
%			    (i : Item v v) ->
%			    i ≋ initial x -> eq-prop (_∈ Sₙ ω) i z
%			  ) ->
%			  Complete (parse₀ ω)
%			complete-parse₀ {a₀} {t} {v = ε} k ω c fx fx₂ = complete-pred-comp k ω c fx fx₂
%			complete-parse₀ {a₀} {t} {v = a ∷ v} k ω c fx fx₂ =
%			  let
%			    x₁ = complete-pred-comp k ω c fx fx₂
%			    x₂ = case k of
%			      λ { (r refl) → l (σ ε refl)
%			        ; (l (σ p₁ p₀)) → l (σ (p₁ ←∷ a₀) (trans (sym (in₀ _ _ _)) (sym p₀)))
%			        }
%			  in
%			  complete-parse₀ {v = v} x₂ (step ω) x₁
%			    (λ {g i refl → complete-scanner a (pred-comp ω) x₁
%			      (_ ∘ _ ↦ _ ∘ _ [ v-unstep (Item.χ i) ∘ Item.ψ i ]) i g refl refl})
%			    (λ {refl x₂ i x₃ → case k of
%			      λ { (r ())
%			        ; (l (σ p₁ p₀)) → void (ε.ε₂ decidₜ (trans (sym (in₀ _ _ _)) (sym p₀)))
%			        }
%			      })
%
%			complete₀-itemize : ∀ w {β} ->
%			  (rs : (Σ λ t -> (t ∈ CFG.rules G) × (fst t ≡ CFG.start G)) *) ->
%			  (x : (CFG.start G , β) ∈ CFG.rules G) ->
%			  (CFG.start G , β) ∈ map Σ.proj₁ rs ->
%			  (i : Item w w) ->
%			  i ≋ initial x ->
%			    i ∈ itemize w rs
%			complete₀-itemize w ε x () i refl
%			complete₀-itemize w (σ (X , β) p₀ ∷ rs) x in-head i refl = in-head
%			complete₀-itemize w (σ (X , β) p₀ ∷ rs) x (in-tail p) i refl =
%			  in-tail (complete₀-itemize w rs x p i refl)
%
%			complete-itemize : ∀ w {β} ->
%			  (x : (CFG.start G , β) ∈ CFG.rules G) ->
%			  (i : Item w w) ->
%			  i ≋ initial x ->
%			    i ∈ itemize w (lookup (CFG.start G) (CFG.rules G))
%			complete-itemize w x i refl =
%			  let x₁ = Σ.proj₀ (lookup-sound x) in
%			  complete₀-itemize w (lookup _ _) x (in-map Σ.proj₁ x₁) i refl
%
%			complete-parse : ∀ a₀ w ->
%			  Complete (parse w)
%			complete-parse a₀ ε =
%			  complete-parse₀ {a = a₀} (r refl) (start (itemize ε (lookup _ _))) top
%			    (λ {g i x → void (test₃ (suff-g₂ g))})
%			    (λ {refl x i x₁ → complete-itemize ε x i x₁})
%			complete-parse a₀ (x ∷ w) =
%			  let
%			    x₁ = start (itemize (x ∷ w) (lookup _ _))
%			    x₂ = complete-pred-comp {a = a₀} (r refl) x₁ top
%			      (λ {g i x₂ → void (test₃ (suff-g₂ g))})
%			      (λ {refl x₂ i refl → complete-itemize (x ∷ w) x₂ i refl})
%			  in
%			  complete-parse₀ (l (σ ε refl)) (step x₁) x₂
%			    (λ g i x₃ → complete-step {a₀ = x} (r refl) x₁ top
%			      (λ {g₁ i₁ x₄ → void (test₃ (suff-g₂ g₁))})
%			      (λ {refl x₄ i₁ x₅ → complete-itemize (x ∷ w) x₄ i₁ x₅})
%			      (_ ∘ _ ↦ _ ∘ _ [ in-g g ∘ suff-g₁ g ]) i g refl x₃)
%			    (λ ())
%
%		\end{code}

	\section{Parse trees}

