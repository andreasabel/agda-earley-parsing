
\chapter{Correctness proofs} \label{Correctness}

	Here, we discuss the structure of the correctness proofs for the
	implementation discussed in Chapter~\ref{Earleys}. First, we discuss
	soundness of the generated item sets, and continue from there to their
	completeness. Finally backtracking from the item sets to create parse
	trees, and ambiguous parses are discussed.

	\section{Soundness}

		First, we provide a definition of sound items sets:

		\begin{code}
			Valid : ∀ {w v} -> Item w v -> Set
			Valid {w} {v} i = G ⊢ Item.u i / v ⟶* Item.Y i / Item.β i

			Sound : ∀ {w v} -> EState w v -> Set
			Sound (start rs) = ∀ {i} -> i ∈ rs -> Valid i
			Sound (step ω rs) = Sound ω × (∀ {i} -> i ∈ rs -> Valid i)
		\end{code}

		An item is \codett{Valid} if it can be derived using the Earley rules,
		and a set of item sets is \codett{Sound} if all items in it are valid.
		This definition of soundness is both simple and intuitive. Proving that
		the implementation always preserves these properties is also very
		straightforward, as the algorithm never modifies items that are added
		to an item set, and the implementation generates new items in three
		different ways, matching the \emph{scan}, \emph{predict}, and
		\emph{complete} steps in the algorithm also present in our Earley
		propositions.

		First, we will take a look at the scanning step. We will consider
		\codett{scanr₀} to be sound if all its generated items are valid, given
		valid input. This allows us to show that \codett{scanr} preserves
		soundness for \codett{EState}s:

		\begin{code}
			sound-scanr₀ : ∀ {a v w} -> ∀ rs ->
			  (∀ {i} -> i ∈ rs -> Valid i) ->
			  (∀ {i} -> i ∈ scanr₀ {w} {v} a rs -> Valid i)

			sound-scanr₀ ε f ()

			sound-scanr₀ ((X ∘ u ↦ α ∘ ε) ∷ rs) f p =
			  sound-scanr₀ rs (f ∘ in-tail) p

			sound-scanr₀ ((X ∘ u ↦ α ∘ l Y ∷ β) ∷ rs) f p =
			  sound-scanr₀ rs (f ∘ in-tail) p

			sound-scanr₀ {a} ((X ∘ u ↦ α ∘ r b ∷ β) ∷ rs) f p with decidₜ a b
			... | no x = sound-scanr₀ rs (f ∘ in-tail) p
			... | yes refl with p
			...            | in-head    = scanner (f in-head)
			...            | in-tail p₁ = sound-scanr₀ rs (f ∘ in-tail) p₁

			sound-scanr : ∀ {a w v} -> (ω : EState w (a ∷ v)) ->
			  Sound ω -> Sound (scanr a ω)
			sound-scanr (start rs) s = s , sound-scanr₀ rs s
			sound-scanr (step w rs) (s , s₁) = s , s₁ , sound-scanr₀ rs s₁
		\end{code}

		The structure of the proof matches the structure of the original
		definitions closely, due to the way normalizaitons are axiomatic to the
		type system. If there is no input, there is no output, so nothing needs
		to be shown. When there is, \codett{scanr₀} only generates new items
		from items that expect a terminal symbol, and only does so when that
		terminal symbol equals the next token in the input sequence. Finally,
		the proof is structured so that an arbitrary generated item is chosen,
		and that item should then be shown to be valid. The argument \codett{p}
		specifies for which of the generated items the proof is desired, and if
		the one generated at this recursion level is not the correct one, we
		must continue the proof search.

		Next, we continue in a similar fashion for the complete step:

		\begin{code}
			sound-compl₀ : ∀ {u v w} (w : EState w v) ->
			  Sound w -> (∀ {i} -> i ∈ compl₀ {u} {v} w -> Valid i)
			sound-compl₀ {u} {v} w s p           with eq-T* u v
			sound-compl₀ {v} {v} (start rs) s p  | yes refl = s
			sound-compl₀ {v} {v} (step w rs) s p | yes refl = snd s
			sound-compl₀ {u} {v} (start rs) s () | no x
			sound-compl₀ {u} {v} (step w rs) s p | no x = sound-compl₀ w (fst s) p
		\end{code}

		As \codett{sound-compl₀} is nothing more than a lookup function to find
		the correct item set to use from a \codett{EState}, its soundness is
		trivial when the soundness of the original item set is given.
		\codett{sound-compl₁} is more interesting:

		\begin{code}
			sound-compl₁ : ∀ {u v w} ->
			  (i : Item w v) ->
			  (p : Item.β i ≡ ε) ->
			  (q : Item.u i ≡ u) ->
			  (rs : Item w u *) ->
			  Valid i -> (∀ {j} -> j ∈ rs -> Valid j) ->
			  (∀ {j} -> j ∈ compl₁ i p rs -> Valid j)

			sound-compl₁ i refl refl ε v f ()

			sound-compl₁ i refl refl ((Y ∘ u₁ ↦ α₁ ∘ ε) ∷ rs) v f q =
			  sound-compl₁ i refl refl rs v (f ∘ in-tail) q

			sound-compl₁ i refl refl ((Y ∘ u₁ ↦ α₁ ∘ r a ∷ β) ∷ rs) v f q =
			  sound-compl₁ i refl refl rs v (f ∘ in-tail) q

			sound-compl₁ i refl refl ((Y ∘ u₁ ↦ α₁ ∘ l Z ∷ β) ∷ rs) v f q with decidₙ X Z
			... | no x = sound-compl₁ i refl refl rs v (f ∘ in-tail) q
			... | yes refl with q =
			...            | in-head     = complet (f in-head) v
			...            | (in-tail q) = sound-compl₁ i refl refl rs v (f ∘ in-tail) q
		\end{code}

		Here, we show, given a valid item that can be completed ($\beta =
		\epsilon$), and an item set from which matching completion items can be
		found, that \codett{compl₁} only generates valid items if all input
		items are valid. Like with the scanning step, we only need to focus on
		the cases where items are actually generated, and when they are, if the
		current one is the one for which a proof is sought. The generated items
		also match the \emph{complete} step of our Earley proposition
		perfectly, so the proof for the relevant item is also very simple.

		The soundness proof for \codett{sound-compl₂} is then trivial:

		\begin{code}
			sound-compl₂ : ∀ {v w} ->
			  (i : Item w v) ->
			  (p : Item.β i ≡ ε) ->
			  Valid i -> (ω : EState w v) -> Sound ω ->
			  (∀ {j} -> j ∈ compl₂ i p ω -> Valid j)
			sound-compl₂ i p v ω s q =
			  sound-compl₁ i p refl (compl₀ ω) v (sound-compl₀ ω s) q
		\end{code}

		As \codett{predict₁} works in very much the same way as
		\codett{compl₂}, both their type proof structure are very similar (the
		only difference in their type being that \codett{compl₂} requires
		$\beta$ to start with a non-terminal instead of being empty), and so
		the soundness proof for \codett{predict₁} will not be discussed in
		further detail.

		Soundness of the merged step \codett{pred-comp} of the \codett{predict}
		and \codett{compl} steps is also completely trivial, as there are no
		modification to existing items nor generation of new ones except for
		what is done in \codett{compl₂} and \codett{predict₁}. Their type
		signatures are shown below:

		\begin{code}
			sound-pred-comp₁ : ∀ {w v} (ω : EState w v) -> ∀ ss rs ->
			  (∀ {i} -> i ∈ ss -> Valid i) ->
			  (∀ {i} -> i ∈ rs -> Valid i) ->
			  Sound ω -> (∀ {i} -> i ∈ pred-comp₁ ω ss rs -> Valid i)
		\end{code}

		The definition of \codett{sound-pred-comp₂}, which does the iterative
		refinement of items for each item set, looks a bit more complex:

		\begin{code}
			sound-pred-comp₂ : ∀ {w v} (ω : EState w v) -> ∀ ss rs m p q ->
			  (∀ {i} -> i ∈ ss -> Valid i) ->
			  (∀ {i} -> i ∈ rs -> Valid i) ->
			  Sound ω -> Sound (pred-comp₂ ω ss rs m p q)
			sound-pred-comp₂ ω ss rs zero () q f g s
			sound-pred-comp₂ ω ss ε (suc m) p q f g s = soundₙ ω f s
			sound-pred-comp₂ ω ss rs@(r₁ ∷ _) (suc m) p q f g s =
			  let x₁ = pred-comp₁ ω ss rs in
			  let x₂ = x₁ \\ (rs ++ ss) in
			  let p₁ = wf-pcw₃ (Σ.proj₀ all-rules) p q  in
			  let p₂ = wf-pcw₂ x₁ (rs ++ ss) q in
			  let p₃ = sound-pred-comp₁ ω ss rs f g s in
			  let p₄ = s-pcw₀ Valid p₃ in
			  sound-pred-comp₂ ω (rs ++ ss) x₂ m p₁ p₂ (in-either Valid g f) p₄ s
		\end{code}

		Fortunately, most of the apparent complexity of this function is caused
		by the termination bounds, which do not directly affect soundness. As
		items are still not modified, we need only to show that the items in
		the item sets after concatenation and de-duplication are indeed the
		same items as they were before, and that all newly generated items are
		\codett{Valid}. The final steps of the parser can be shown to preserve
		soundness in the same way:

		\begin{code}
			sound-pred-comp : ∀ {w v} {ω : EState w v} ->
			  Sound ω -> Sound (pred-comp ω)

			sound-step₀ : ∀ {w a v} {ω : EState w (a ∷ v)} ->
			  Sound ω -> Sound (step₀ ω)

			sound-parse₀ : ∀ {w v} -> {ω : EState w v} ->
			  Sound ω -> Sound (parse₀ ω)

			sound-parse : ∀ w -> Sound (parse w)
		\end{code}

		Which concludes the soundness proof for our Earley implementation.

	\section{Completeness}

		The completeness proof is significantly more lengthy than the soundness
		proof. This is caused by the fact that while soundness is a very
		'local' property, dealing only with individual items, completeness is
		much more global in nature, dealing with the entire item sets. This
		means all operations that could previously be glanced over will now
		require more extensive analysis.

		We will begin with a definition of completeness for item sets:

		\begin{code}
			_≋_ : ∀ {t u v X α β} -> Item t v -> G ∙ t ⊢ u / v ⟶* X / α ∙ β -> Set
			_≋_ {t} {u} {v} {X} {α} {β} i g =
				(Item.Y i , Item.u i , Item.α i , Item.β i) ≡ (X , u , α , β)

			Complete₀ : ∀ {t v} -> Item t v * -> Set
			Complete₀ {t} {v} rs = ∀ {u Y α β} ->
			  (i : Item t v) ->
			  (g : G ∙ t ⊢ u / v ⟶* Y / α ∙ β) ->
			  i ≋ g ->
			  i ∈ rs
		\end{code}

		First a bit of notation: we use \codett{i ≋ g} to mean that an item $i$
		is equal to the item that should be present in the item set given the
		Earley proposition $g$ defined earlier in this chapter. A set of items
		that represent partial parses of an input string $t$ before parsing
		remainder $v$ is then considered complete if any items that represent
		any possible Earley state in the item set between the last element of
		$t$ and the first of $u$ is present in the set. This is a 'shallow'
		definition of completeness: only the topmost item set is considered,
		with all prevous ones disregarded. We also define two 'deeper' notions of
		completeness for \codett{EState}s:

		\begin{code}
			mutual
			  Complete : ∀ {v w} -> EState w v -> Set
			  Complete ω = Complete₀ (Sₙ ω) × Complete* ω

			  Complete* : ∀ {v w} -> EState w v -> Set
			  Complete* (start rs) = ⊤
			  Complete* (step ω rs) = Complete ω
		\end{code}

		Here, the \codett{Complete*} property for a \codett{EState} would mean
		that all item sets, except for the very topmost one, are complete, and
		\codett{Complete} represents completeness for \codett{EState}s: all
		prevous item sets are complete, as is the topmost one.

		We start by taking a fairly high-level overview of the completeness
		proof which we will do by induction on the Earley propositions. The
		type for this function is fairly lengthy, so we will break it up into
		several peices and discuss them separately:

		\begin{code}
			complete-ind : ∀ {t v a} ->
			  (Σ λ u -> u ++ (a ∷ v) ≡ t) ∣ v ≡ t ->
			  {P : Item t v -> Set} ->

			  (∀ {u X α β} ->
			    (g : G ∙ t ⊢ u / a ∷ v ⟶* X / α ∙ r a ∷ β) ->
			    (i : Item t v) -> (i ≋ scanner g) ->
			    P i
			  ) ->

			  (∀ {β} ->
			    (z : t ≡ v) ->
			    (x : (CFG.start G , β) ∈ CFG.rules G) ->
			    (i : Item v v) ->
			    i ≋ initial x -> eq-prop P i z
			  ) ->
		\end{code}

		To begin with, the original input string $t$, and its current remainder
		$v$ are given as implicit arguments, but together with these we also
		provide a single token that is used to name the last scanned token,
		together with a proof that this actually was the previous token.  The
		requirement of this token as an argument restricts this proof to only
		be valid for parsing alphabets that have at least one token, but as
		token-free parsing is completely trivial we did not feel compelled to
		remove this restriction. This previous token is used in the first
		function argument, which represents an assumption that the property
		\codett{P} holds for all items that can be generated from the scanning
		step. The second argument represents the same assumption, but for the
		initial items. This will only be relevant when we are considering the
		very first item set, which is why \codett{t ≡ v}. Here,
		\codett{eq-prop} is only present because of the slight mismatch of
		types between the initial items and \codett{P} caused by this equality
		constraint, and can just be thought of as \codett{P i}. At this time,
		we do not need to bother with what specific proposition we will use
		this function to prove, which is why it is parameterized on the
		proposition \codett{P}.

		Continuing on, we have:

		\begin{code}
			  (∀ {u X Y α β δ} (x : (Y , δ) ∈ CFG.rules G) ->
			    (i j : Item t v) ->
			    (g : G ∙ t ⊢ u / v ⟶* X / α ∙ l Y ∷ β) ->
			    (i ≋ g) -> P i ->
			    (j ≋ predict x g) -> P j
			  ) ->

			  (∀ {u w X Y α β γ} ->
			    (j k : Item t v) ->
			    (g : G ∙ t ⊢ u / w ⟶* X / α ∙ l Y ∷ β) ->
			    (h : G ∙ t ⊢ w / v ⟶* Y / γ ∙ ε) ->
			    (j ≋ h) -> P j ->
			    (k ≋ complet g h) -> P k
			  ) ->
		\end{code}

		These arguments fill a similar function as the previous two, but for
		items that can be generated via the predict and complete steps,
		repspectively. While their function as part of the induction step is
		equivalent to that of the two previous arguments, when making use of
		this induction step these will represent the main reasoning of the
		proof.

		\begin{code}
			  (∀ {u X α β} ->
			    (g : G ∙ t ⊢ u / v ⟶* X / α ∙ β) ->
			    (i : Item t v) ->
			    i ≋ g ->
			    P i
			  )

			complete-ind s f ini h c (initial x) i refl =
			  ini refl x i refl

			complete-ind {a} s f ini h c (scanner g) i p =
			  case test₀ s g of λ {refl -> f g i p}

			complete-ind s f ini h c (predict x g) i p =
			  let x₁ = complete-ind s f ini h c g _ refl in
			  h x (_ ∘ _ ↦ _ ∘ _ [ in-g g ∘ suff-g₁ g ]) i g refl x₁ p

			complete-ind s f ini h c (complet g g₁) i p =
			  let x₁ = complete-ind s f ini h c g₁ _ refl in
			  c (_ ∘ _ ↦ _ ∘ _ [ in-g g₁ ∘ suff-g₂ g ]) i g g₁ refl x₁ p
		\end{code}

		Given that the proposition holds for all ways of counstructing items in
		ways that satisfy the Earley rules discuss in Chapter~\ref{Earleys}, it
		will hold for all items that can be constructed using these rules. The
		proof is straightforward, but unfortunately, while the compiler can
		automatically infer the item needed in the recursion, it is unable to
		infer it's $\chi$ and $\psi$ values. Fortunately, these can be
		recunstructed from the Earley rules.

		In order to prove completeness for the entire algorithm, we will now
		use \codett{complete-ind} to show that all items in the current item
		set will be generated by \codett{pred-comp} and use this together with
		completness for \codett{scan} to show that the generation of the items
		in each item set is complete. This can then be applied recursively to
		show that the entire \codett{EState} is complete. Of these steps, the
		most complex is completeness for \codett{pred-comp}, which needs to be
		done in two steps: completeness for items that are \codett{predict}ed,
		and completeness for items that are \codett{compl}eted. This is further
		complicated by the fact that some \codett{compl}eted items may in fact
		be generated by the \codett{predict} stage instead of the
		\codett{compl} stage. This happens when there are rules which accept
		the empty string, which the \codett{compl} stage is unable to handle
		properly. The combined predicted and completed item is then added
		directly by the predict stage, similarly to what is done in
		\cite{aycock02}.

		To show completeness for the \codett{predict}ed items from
		\codett{pred-comp₂} we will make use of the fact that items that have
		already been predicted on cannot be combined in any other way to form
		new items that are different than the ones created by prediction the
		first time. This is represented in the proof as the set \codett{ss}
		being \codett{Inert}.

		\begin{code}
			Inert : ∀ {t v} -> Item t v * -> Item t v * -> Set
			Inert {t} {v} ss rs =
			  ∀ {u X Y α β δ} ->
			  (x : (Y , δ) ∈ CFG.rules G) ->
			  (i j : Item t v) ->
			  (g : G ∙ t ⊢ u / v ⟶* X / α ∙ l Y ∷ β) ->
			  i ≋ g -> i ∈ ss ->
			  j ≋ predict x g ->
			    j ∈ (rs ++ ss)
		\end{code}

		\codett{ss} is inert if all items derivable form it are either in
		\codett{ss} or another specified set \codett{rs}: the items next in
		line to be predicted and completed on. We can then show that the basic
		step in \codett{pred-comp₂} (predicting and completeing on \codett{rs}
		to form the next \codett{rs}, and adding \codett{rs} to the inert set)
		preserves inertness:

		\begin{code}
			inert' : ∀ {t v rs ss} (ω : EState t v) ->
			  Inert ss rs -> Inert (rs ++ ss) (pred-comp₁ ω ss rs \\ (rs ++ ss))
		\end{code}

		This then makes completeness for \codett{pred-comp₂} predicted items
		simple to prove. Only some book-keeping related to the termination of
		\codett{pred-comp₂} is needed in addition to \codett{inert'}, and so
		only the type of the proof is shown here:

		\begin{code}
			complete₁-pred-comp₂ : ∀ {t v} -> ∀ ss rs m p q ->
			  (ω : EState t v) ->
			  Inert ss rs ->
			  ∀ {u X Y α β δ}
			  (x : (Y , δ) ∈ CFG.rules G) ->
			  (i j : Item t v) ->
			  (g : G ∙ t ⊢ u / v ⟶* X / α ∙ l Y ∷ β) ->
			  i ≋ g ->
			  i ∈ Sₙ (pred-comp₂ ω ss rs m p q) ->
			  j ≋ predict x g ->
			    j ∈ Sₙ (pred-comp₂ ω ss rs m p q)
		\end{code}

		This is done in the same way for completed items, mirroring the
		arguments to \codett{complete-ind}:

		\begin{code}
			Inert₂ : ∀ {t v} -> Item t v * -> Item t v * -> Set
			Inert₂ {t} {v} ss rs =
			  ∀ {u w X Y α β γ} ->
			  (j k : Item t v) ->
			  (g : G ∙ t ⊢ u / w ⟶* X / α ∙ l Y ∷ β) ->
			  (h : G ∙ t ⊢ w / v ⟶* Y / γ ∙ ε) ->
			  j ≋ h -> j ∈ ss ->
			  k ≋ complet g h ->
			    k ∈ (rs ++ ss)
		\end{code}

		However, \codett{inert₂'} is complicated somewhat by the need to
		differentiate between items that where completed using the
		\codett{compl} step, and the ones that were completed immediately opon
		prediction:

		\begin{code}
			inert₂' : ∀ {t v rs ss} (ω : EState t v) ->
			  Complete* ω ->
			  Inert₂ ss rs ->
			  Inert₂ (rs ++ ss) (pred-comp₁ ω ss rs \\ (rs ++ ss))
			inert₂' {rs = ε} {ss} ω c it x i j g z₁ z₂ z₃ = it x i j g z₁ z₂ z₃
			inert₂' {rs = rs@(r₁ ∷ rs₀)} {ss} ω c it x i j g z₁ z₂ z₃ =
			  case in-lr rs ss z₂ of
			    λ { (r x₁) →
			      let x₁ = it x i j g z₁ x₁ z₃  in
			      in-r x₁
			    ; (l x₁) → empty-rule? x i j g z₁ x₁ z₃
			    }
		\end{code}

		When this has been done, completeness for the items that have been
		completed from \codett{pred-comp₂} is simple to prove, with only some
		extra book-keeping being nessacary.

		\begin{code}
			complete₂-pred-comp₂ : ∀ {t v} -> ∀ ss rs m p q ->
			  (ω : EState t v) ->
			  Complete* ω ->
			  Inert₂ ss rs ->
			  ∀ {u w X Y α β γ}
			  (j k : Item t v) ->
			  (g : G ∙ t ⊢ u / w ⟶* X / α ∙ l Y ∷ β) ->
			  (h : G ∙ t ⊢ w / v ⟶* Y / γ ∙ ε) ->
			  j ≋ h -> j ∈ Sₙ (pred-comp₂ ω ss rs m p q) ->
			  k ≋ complet g h ->
			    k ∈ Sₙ (pred-comp₂ ω ss rs m p q)
		\end{code}

		What remains to be shown is then that the scanning step is complete:

		\begin{code}
			complete-scanner : ∀ {t u v X α β} -> ∀ a ->
			  (ω : EState t (a ∷ v)) ->
			  Complete ω ->
			  (i : Item t (a ∷ v)) ->
			  (j : Item t v) ->
			  (g : G ∙ t ⊢ u / a ∷ v ⟶* X / α ∙ r a ∷ β) ->
			  i ≋ g ->
			  j ≋ scanner g ->
			    j ∈ Sₙ (scanner a ω)
		\end{code}

		Followed by a proof that all initial items are present (should the
		current item set be the first one):

		\begin{code}
			complete-itemize : ∀ w {β} ->
			  (x : (CFG.start G , β) ∈ CFG.rules G) ->
			  (i : Item w w) ->
			  i ≋ initial x ->
			    i ∈ itemize w (lookup (CFG.start G) (CFG.rules G))
		\end{code}

		The \codett{step} function will then preserve completeness when
		generating new item sets:

		\begin{code}
			complete-step : ∀ {a₀ a t v} ->
			  (Σ λ u -> u ++ (a₀ ∷ a ∷ v) ≡ t) ∣ (a ∷ v) ≡ t ->
			  (ω : EState t (a ∷ v)) ->
			  Complete* ω ->
			  (∀ {u X α β} ->
			    (g : G ∙ t ⊢ u / a₀ ∷ a ∷ v ⟶* X / α ∙ r a₀ ∷ β) ->
			    (i : Item t (a ∷ v)) -> (i ≋ scanner g) ->
			    i ∈ Sₙ ω
			  ) ->
			  (∀ {β} ->
			    (z : t ≡ a ∷ v) ->
			    (x : (CFG.start G , β) ∈ CFG.rules G) ->
			    (i : Item (a ∷ v) (a ∷ v)) ->
			    i ≋ initial x -> eq-prop (_∈ Sₙ ω) i z
			  ) ->
			  ∀ {u X α β} ->
			  (i : Item t (a ∷ v)) ->
			  (j : Item t v) ->
			  (g : G ∙ t ⊢ u / a ∷ v ⟶* X / α ∙ r a ∷ β) ->
			  i ≋ g ->
			  j ≋ scanner g ->
			    j ∈ Sₙ (step ω)
			complete-step {a₀} {a} s ω c fx fx₂ i j g refl refl =
			  let
			    x₁ = complete-pred-comp s ω c fx fx₂
			  in complete-scanner a (pred-comp ω) x₁ i j g refl refl
		\end{code}

		This can then be used to show that any number of steps still preserve
		completeness:

		\begin{code}
			complete-parse₀ : ∀ {a t v} ->
			  (Σ λ u -> u ++ (a ∷ v) ≡ t) ∣ v ≡ t ->
			  (ω : EState t v) ->
			  Complete* ω ->
			  (∀ {u X α β} ->
			    (g : G ∙ t ⊢ u / a ∷ v ⟶* X / α ∙ r a ∷ β) ->
			    (i : Item t v) -> (i ≋ scanner g) ->
			    i ∈ Sₙ ω
			  ) ->
			  (∀ {β} ->
			    (z : t ≡ v) ->
			    (x : (CFG.start G , β) ∈ CFG.rules G) ->
			    (i : Item v v) ->
			    i ≋ initial x -> eq-prop (_∈ Sₙ ω) i z
			  ) ->
			  Complete (parse₀ ω)
			complete-parse₀ {a₀} {t} {v = ε} k ω c fx fx₂ = complete-pred-comp k ω c fx fx₂
			complete-parse₀ {a₀} {t} {v = a ∷ v} k ω c fx fx₂ =
			  let
			    x₁ = complete-pred-comp k ω c fx fx₂
			    x₂ = case k of
			      λ { (r refl) → l (σ ε refl)
			        ; (l (σ p₁ p₀)) → l (σ (p₁ ←∷ a₀) (trans (sym (in₀ _ _ _)) (sym p₀)))
			        }
			  in
			  complete-parse₀ {v = v} x₂ (step ω) x₁
			    (λ {g i refl → complete-scanner a (pred-comp ω) x₁
			      (_ ∘ _ ↦ _ ∘ _ [ v-unstep (Item.χ i) ∘ Item.ψ i ]) i g refl refl})
			    (λ {refl x₂ i x₃ → case k of
			      λ { (r ())
			        ; (l (σ p₁ p₀)) → void (ε.ε₂ decidₜ (trans (sym (in₀ _ _ _)) (sym p₀)))
			        }
			      })
		\end{code}

		Supplying the proof that the initial states are complete we can finally
		show completeness of the \codett{parse} function:

		\begin{code}
			complete-parse : ∀ a₀ w ->
			  Complete (parse w)
			complete-parse a₀ ε =
			  complete-parse₀ {a = a₀} (r refl) (start (itemize ε (lookup _ _))) top
			    (λ {g i x → void (test₃ (suff-g₂ g))})
			    (λ {refl x i x₁ → complete-itemize ε x i x₁})
			complete-parse a₀ (x ∷ w) =
			  let
			    x₁ = start (itemize (x ∷ w) (lookup _ _))
			    x₂ = complete-pred-comp {a = a₀} (r refl) x₁ top
			      (λ {g i x₂ → void (test₃ (suff-g₂ g))})
			      (λ {refl x₂ i refl → complete-itemize (x ∷ w) x₂ i refl})
			  in
			  complete-parse₀ (l (σ ε refl)) (step x₁) x₂
			    (λ g i x₃ → complete-step {a₀ = x} (r refl) x₁ top
			      (λ {g₁ i₁ x₄ → void (test₃ (suff-g₂ g₁))})
			      (λ {refl x₄ i₁ x₅ → complete-itemize (x ∷ w) x₄ i₁ x₅})
			      (_ ∘ _ ↦ _ ∘ _ [ in-g g ∘ suff-g₁ g ]) i g refl x₃)
			    (λ ())
		\end{code}

		We can here again see the argument \codett{a₀} which represents the
		previous token in \codett{complete-ind} causing a requirement of
		providing a token to the final completeness proof. Any token will work,
		however, and so this only creates a requirement that at least one token
		exists. Grammars over languages with no symbols can only accept or
		reject the empty string (as no other input sequence could be
		constructed) and as such they are not very interesting, and we do not
		feel that this restriction is especially problematic. This concludes
		the completeness proof for the implemented parser.

	\section{Parse trees}

		The output of the of the implemented parser - an \codett{EState}
		essentially being a set of set of Earley items - is decidedly not a
		parse tree.  Parse tree are not generally useful when parsing
		potentially ambiguous grammars, as the ambiguities can create an
		exponential blowup in the number of nodes in the resulting parse tree,
		or even create potentially infinite loops of ambiguities. Depending on
		how the result will be used different structures might be appropriate
		such as Sparse Packed Parse Forests~\cite{?}, or to simply always
		choose the first option when an ambiguity arises.
