\AgdaHide{\begin{code}
module Slides (N T : Set) where
\end{code}}

\begin{frame}
	\frametitle{What's Agda?}

	\begin{itemize}
		\item Functional programming language
		\item Haskell-like \emph{(ish)} syntax
		\item Dependently typed
		\item Dependent types allow expressing constructing proofs as terms
			in the language
	\end{itemize}
\end{frame}

\begin{frame}
	\frametitle{What's Agda?}
	% Code example (++-assoc from report?) as quick intro to agda
	\begin{code}
		data ℕ : Set where
		  zero  :  ℕ
		  suc   :  ℕ → ℕ
	\end{code}
	\vspace{0.5cm}
	\begin{code}
		_+_ : ℕ → ℕ → ℕ
		zero   + b = b
		suc a  + b = suc (a + b)
	\end{code}
\end{frame}

\begin{frame}
	\frametitle{What's Agda?}
	\begin{code}
		data _≤_ : ℕ → ℕ → Set where
		  ≤₀  : {n : ℕ} → zero ≤ n
		  ≤ₛ  : {m n : ℕ} → m ≤ n → suc m ≤ suc n
	\end{code}
\end{frame}

\AgdaHide{
\begin{code}
module Slides-2 where
open import base

record CFG : Set where
  constructor _⟩_
  field
    start : N
    rules : (N × (N ∣ T)* )*
\end{code} }


\begin{frame}
	\frametitle{What's Parsing?}
	% Short intro to parsing, what is a cfg, what do are ambiguous things

	\begin{itemize*}
		\item Analyze the structure of a sequence of symbols
		\item Structures are called grammars
		\item We're interested in context-free grammars
	\end{itemize*}\\
	\vspace{0.5cm}
	\centering
	\begin{tabular}{c}
		$X \mapsto \alpha$
	\end{tabular}
	\vspace{2cm}

\end{frame}

\begin{frame}
	\frametitle{What's Parsing?}
	% Short intro to parsing, what is a cfg, what do are ambiguous things

	\begin{itemize*}
		\item Analyze the structure of a sequence of symbols
		\item Structures are called grammars
		\item We're interested in context-free grammars
	\end{itemize*}\\
	\vspace{0.5cm}
	\centering
	\begin{tabular}{rcl}
		$E$ & $ \mapsto $ & $V$ \\
		$E$ & $ \mapsto $ & $E$ $O$ $E$ \\
		$O$ & $ \mapsto $ & `$+$' \\
		$O$ & $ \mapsto $ & `$-$' \\
		$V$ & $ \mapsto $ & `$x$' \\
		$V$ & $ \mapsto $ & `$y$' \\
		$V$ & $ \mapsto $ & `$z$'
	\end{tabular}
\end{frame}

\begin{frame}
	\centering
	\begin{tabular}{rclcrclcrcl}
		$E$ & $ \mapsto $ & $V$         && $O$ & $ \mapsto $ & `$+$' && $V$ & $ \mapsto $ & `$x$' \\
		$E$ & $ \mapsto $ & $E$ $O$ $E$ && $O$ & $ \mapsto $ & `$-$' && $V$ & $ \mapsto $ & `$y$' \\
		    &             &             &&     &             &       && $V$ & $ \mapsto $ & `$z$'
	\end{tabular}
\end{frame}

\begin{frame}
	\begin{code}
		infixr 10 _⊢_∈_
		data _⊢_∈_ (G : CFG) :  T * → (N ∣ T) * → Set where
		  empty :
		    G ⊢ ε ∈ ε

		  concat : {u v : T *} {A : N} {α : (N ∣ T) *} →
		    G ⊢ u ∈ l A ∷ ε → G ⊢ v ∈ α → G ⊢ u ++ v ∈ l A ∷ α

		  term : {u : T *} {a : T} {α : (N ∣ T) *} →
		    G ⊢ u ∈ α → G ⊢ a ∷ u ∈ r a ∷ α

		  nonterm : {A : N} {α : (N ∣ T) *} {u : T *} →
		    (A , α) ∈ CFG.rules G → G ⊢ u ∈ α → G ⊢ u ∈ l A ∷ ε
	\end{code}
\end{frame}

\begin{frame}
	\frametitle{What's Earley?}
	% Walk through parsing a string using Earley's Algorithm. All
	% explanation done during parsing

	\begin{itemize}
		\item The Earley parsing algorithm. Named after inventor, Jay Earley
		\item Fast for some types of grammars
		\item Not previously mechanically verified
		\item Table parser
		\item Items on the form $\earley{X}{\alpha}{\beta}{i}$
	\end{itemize}
\end{frame}

\begin{frame}
	\frametitle{Earley's Algorithm}
	\centering
	\begin{tabular}{rcl}
		$E$ & $ \mapsto $ & $V$ \\
		$E$ & $ \mapsto $ & $E$ $O$ $E$ \\
		$O$ & $ \mapsto $ & `$+$' \\
		$O$ & $ \mapsto $ & `$-$' \\
		$V$ & $ \mapsto $ & `$x$' \\
		$V$ & $ \mapsto $ & `$y$' \\
		$V$ & $ \mapsto $ & `$z$'
	\end{tabular} \\ \vspace{1cm}
	% This table, or parts of it should be part of several (all?) of the
	% slides where the earley states are shown.
	Parse the string ``\emph{x + y}''
\end{frame}

\begin{frame}
	\frametitle{Earley's Algorithm}
	\centering
	\begin{tabular}{|c|c|c|c|}
		\hline
		$\St{0}$ & $\St{1}$ & $\St{2}$ & $\St{3}$ \\
		\hline
		$\earley{S_0}{\varepsilon}{E}{0}$ & & & \\
		\hline
	\end{tabular}

	\\~\\
	\vspace{0.5cm}
	\scriptsize
	\begin{tabular}{rclcrclcrcl}
		$E$ & $ \mapsto $ & $V$         && $O$ & $ \mapsto $ & `$+$' && $V$ & $ \mapsto $ & `$x$' \\
		$E$ & $ \mapsto $ & $E$ $O$ $E$ && $O$ & $ \mapsto $ & `$-$' && $V$ & $ \mapsto $ & `$y$' \\
		    &             &             &&     &             &       && $V$ & $ \mapsto $ & `$z$'
	\end{tabular}
\end{frame}

\begin{frame}
	\frametitle{Earley's Algorithm}
	\centering
	\begin{tabular}{|c|c|c|c|}
		\hline
		$\St{0}$ & $\St{1}$ & $\St{2}$ & $\St{3}$ \\
		\hline
		$\earley{S_0}{\varepsilon}{E}{0}$ & & & \\
		$\earley{E}{\varepsilon}{V}{0}$   & & & \\
		$\earley{E}{\varepsilon}{EOE}{0}$ & & & \\
		\hline
	\end{tabular}

	\\~\\
	\vspace{0.5cm}
	\scriptsize
	\begin{tabular}{rclcrclcrcl}
		$E$ & $ \mapsto $ & $V$         && $O$ & $ \mapsto $ & `$+$' && $V$ & $ \mapsto $ & `$x$' \\
		$E$ & $ \mapsto $ & $E$ $O$ $E$ && $O$ & $ \mapsto $ & `$-$' && $V$ & $ \mapsto $ & `$y$' \\
		    &             &             &&     &             &       && $V$ & $ \mapsto $ & `$z$'
	\end{tabular}
\end{frame}

\begin{frame}
	\frametitle{Earley's Algorithm}
	\centering
	\begin{tabular}{|c|c|c|c|}
		\hline
		$\St{0}$                          & $\St{1}$ & $\St{2}$ & $\St{3}$ \\
		\hline
		$\earley{S_0}{\varepsilon}{E}{0}$ & $\earley{V}{x}{\varepsilon}{0}$ & &\\
		$\earley{E}{\varepsilon}{V}{0}$   & & & \\
		$\earley{E}{\varepsilon}{EOE}{0}$ & & & \\
		$\earley{V}{\varepsilon}{x}{0}$   & & & \\
		$\earley{V}{\varepsilon}{y}{0}$   & & & \\
		$\earley{V}{\varepsilon}{z}{0}$   & & & \\
		\hline
	\end{tabular}

	\\~\\
	\vspace{0.5cm}
	\scriptsize
	\begin{tabular}{rclcrclcrcl}
		$E$ & $ \mapsto $ & $V$         && $O$ & $ \mapsto $ & `$+$' && $V$ & $ \mapsto $ & `$x$' \\
		$E$ & $ \mapsto $ & $E$ $O$ $E$ && $O$ & $ \mapsto $ & `$-$' && $V$ & $ \mapsto $ & `$y$' \\
		    &             &             &&     &             &       && $V$ & $ \mapsto $ & `$z$'
	\end{tabular}
\end{frame}

\begin{frame}
	\frametitle{Earley's Algorithm}
	\centering
	\begin{tabular}{|c|c|c|c|}
		\hline
		$\St{0}$                          & $\St{1}$ & $\St{2}$ & $\St{3}$ \\
		\hline
		$\earley{S_0}{\varepsilon}{E}{0}$ & $\earley{V}{x}{\varepsilon}{0}$   &  & \\
		$\earley{E}{\varepsilon}{V}{0}$   & $\earley{E}{V}{\varepsilon}{0}$   &  & \\
		$\earley{E}{\varepsilon}{EOE}{0}$ & $\earley{S_0}{E}{\varepsilon}{0}$ &  & \\
		$\earley{V}{\varepsilon}{x}{0}$   & $\earley{E}{E}{OE}{0}$            &  & \\
		$\earley{V}{\varepsilon}{y}{0}$   &                                   &  & \\
		$\earley{V}{\varepsilon}{z}{0}$   &                                   &  & \\
		\hline
	\end{tabular}

	\\~\\
	\vspace{0.5cm}
	\scriptsize
	\begin{tabular}{rclcrclcrcl}
		$E$ & $ \mapsto $ & $V$         && $O$ & $ \mapsto $ & `$+$' && $V$ & $ \mapsto $ & `$x$' \\
		$E$ & $ \mapsto $ & $E$ $O$ $E$ && $O$ & $ \mapsto $ & `$-$' && $V$ & $ \mapsto $ & `$y$' \\
		    &             &             &&     &             &       && $V$ & $ \mapsto $ & `$z$'
	\end{tabular}
\end{frame}

\begin{frame}
	\frametitle{Earley's Algorithm}
	\centering
	\begin{tabular}{|c|c|c|c|}
		\hline
		$\St{0}$                          & $\St{1}$                          &\St{2} & $\St{3}$ \\
		\hline
		$\earley{S_0}{\varepsilon}{E}{0}$ & $\earley{V}{x}{\varepsilon}{0}$   & $\earley{O}{+}{\varepsilon}{1}$ & \\
		$\earley{E}{\varepsilon}{V}{0}$   & $\earley{E}{V}{\varepsilon}{0}$   & & \\
		$\earley{E}{\varepsilon}{EOE}{0}$ & $\earley{S_0}{E}{\varepsilon}{0}$ & & \\
		$\earley{V}{\varepsilon}{x}{0}$   & $\earley{E}{E}{OE}{0}$            & & \\
		$\earley{V}{\varepsilon}{y}{0}$   & $\earley{O}{\varepsilon}{+}{0}$   & & \\
		$\earley{V}{\varepsilon}{z}{0}$   & $\earley{O}{\varepsilon}{-}{0}$   & & \\
		\hline
	\end{tabular}

	\\~\\
	\vspace{0.5cm}
	\scriptsize
	\begin{tabular}{rclcrclcrcl}
		$E$ & $ \mapsto $ & $V$         && $O$ & $ \mapsto $ & `$+$' && $V$ & $ \mapsto $ & `$x$' \\
		$E$ & $ \mapsto $ & $E$ $O$ $E$ && $O$ & $ \mapsto $ & `$-$' && $V$ & $ \mapsto $ & `$y$' \\
		    &             &             &&     &             &       && $V$ & $ \mapsto $ & `$z$'
	\end{tabular}
\end{frame}

\begin{frame}
	\frametitle{Earley's Algorithm}
	\centering
	\begin{tabular}{|c|c|c|c|}
		\hline
		$\St{0}$                          & $\St{1}$                          & $\St{2}$                         $ & $\St{3}$ \\
		\hline
		$\earley{S_0}{\varepsilon}{E}{0}$ & $\earley{V}{x}{\varepsilon}{0}$   & $\earley{O}{+}{\varepsilon}{1}  $ & $\earley{V}{y}{\varepsilon}{2}  $ \\
		$\earley{E}{\varepsilon}{V}{0}$   & $\earley{E}{V}{\varepsilon}{0}$   & $\earley{E}{EO}{E}{0}           $ & $\earley{E}{V}{\varepsilon}{2}  $ \\
		$\earley{E}{\varepsilon}{EOE}{0}$ & $\earley{S_0}{E}{\varepsilon}{0}$ & $\earley{E}{\varepsilon}{V}{2}  $ & $\earley{E}{EOE}{\varepsilon}{0}$ \\
		$\earley{V}{\varepsilon}{x}{0}$   & $\earley{E}{E}{OE}{0}$            & $\earley{E}{\varepsilon}{EOE}{2}$ & $\earley{E}{E}{OE}{2}           $ \\
		$\earley{V}{\varepsilon}{y}{0}$   & $\earley{O}{\varepsilon}{+}{1}$   & $\earley{V}{\varepsilon}{x}{2}  $ & $\earley{S_0}{E}{\varepsilon}{0}$ \\
		$\earley{V}{\varepsilon}{z}{0}$   & $\earley{O}{\varepsilon}{-}{1}$   & $\earley{V}{\varepsilon}{y}{2}  $ & $\earley{O}{\varepsilon}{+}{3}  $ \\
		$                                 &                                   & $\earley{V}{\varepsilon}{z}{2}  $ & $\earley{O}{\varepsilon}{-}{3}  $ \\
		\hline
	\end{tabular}
	\\~\\
	\vspace{0.5cm}
	\scriptsize
	\begin{tabular}{rclcrclcrcl}
		$E$ & $ \mapsto $ & $V$         && $O$ & $ \mapsto $ & `$+$' && $V$ & $ \mapsto $ & `$x$' \\
		$E$ & $ \mapsto $ & $E$ $O$ $E$ && $O$ & $ \mapsto $ & `$-$' && $V$ & $ \mapsto $ & `$y$' \\
		    &             &             &&     &             &       && $V$ & $ \mapsto $ & `$z$'
	\end{tabular}
\end{frame}

\begin{frame}
	\frametitle{Earley's Algorithm - Formalization}
	\AgdaHide{\begin{code}
		infixl 10 _∙_⊢_/_⟶*_/_∙_
	\end{code}}
	\begin{code}
	data _∙_⊢_/_⟶*_/_∙_ (G : CFG) (input : T *) :
	  (u v : T *) → N → (N ∣ T) * → (N ∣ T) * → Set where
	\end{code}
\end{frame}

\begin{frame}
	\frametitle{Earley's Algorithm - Formalization}
	\begin{code}
	  initial : {α : (N ∣ T) *} →
	    (CFG.start G , α) ∈ CFG.rules G →
	    G ∙ input ⊢ input / input ⟶* CFG.start G / ε ∙ α
	\end{code}
\end{frame}

\begin{frame}
	\frametitle{Earley's Algorithm - Formalization}
	\begin{code}
	  scanner : {u v : T *} {a : T} {X : N} {α β : (N ∣ T) *} →
	    G ∙ input ⊢ u / a ∷ v ⟶* X / α ∙ (r a ∷ β) →
	    G ∙ input ⊢ u / v ⟶* X / (α ←∷ r a) ∙ β

	  predict : {u v : T *} {X Y : N} {α β δ : (N ∣ T) *} →
	    CFG.rules G ∋ (Y , δ) →
	    G ∙ input ⊢ u / v ⟶* X / α ∙ (l Y ∷ β) →
	    G ∙ input ⊢ v / v ⟶* Y / ε ∙ δ

	  complet : {u v w : T *} {X Y : N} {α β γ : (N ∣ T) *} →
	    G ∙ input ⊢ u / v ⟶* X / α ∙ (l Y ∷ β) →
	    G ∙ input ⊢ v / w ⟶* Y / γ ∙ ε →
	    G ∙ input ⊢ u / w ⟶* X / (α ←∷ l Y) ∙ β
	\end{code}
\end{frame}

\begin{frame}
	\frametitle{Earley's Algorithm - Implementation}
	\begin{itemize}
		\item I implemented the stuff we just talked about.
		\item Nullable rules require special care.
		\item Adding more rules to each set while trying to consume it
			complicates termination
	\end{itemize}
\end{frame}

\begin{frame}
	\frametitle{Trying it out}
	% Show the parser parsing some input. Maybe the
	% The only results are accept or reject
	% "equal number of a's and b's" grammar.
\end{frame}

\AgdaHide{\begin{code}
module parser (G : CFG) where

  record Item (w : T *) (v : T *) : Set where
    constructor _∘_↦_∘_
    field
      Y : N
      u : T *
      α β : (N ∣ T) *
      .{χ} : CFG.rules G ∋ (Y , α ++ β)
      .{ψ} : (Σ λ t → t ++ u ≡ w)        -- u is a suffix of w

  data EState : T * → T * → Set where
    start : {v : T *} →
      (rs : Item v v * ) →
      EState v v

    step : {a : T} {w v : T *} →
      (ω : EState w (a ∷ v)) →
      (rs : Item w v * ) →
      EState w v

  _≋_ : T *
  _≋_ = ε

  Latest : T *
  Latest = ε
\end{code}}

\begin{frame}
	\frametitle{Soundness proof}
	% Show definition of soundness
	\begin{code}
		  Valid : ∀ {w v} → Item w v → Set
		  Valid {w} {v} i =
		    G ∙ w ⊢ Item.u i / v ⟶* Item.Y i / Item.α i ∙ Item.β i

		  Sound : ∀ {w v} → EState w v → Set
		  Sound (start rs) = ∀ {i} → i ∈ rs → Valid i
		  Sound (step ω rs) = Sound ω × (∀ {i} → i ∈ rs → Valid i)
  	\end{code}
\end{frame}

\begin{frame}
	\frametitle{Completeness proof}
	% Show definition of completeness
	\begin{code}
		  Complete₀ : ∀ {t v} → Item t v * → Set
		  Complete₀ {t} {v} rs = ∀ {u Y α β} →
		    (i : Item t v) →
		    (g : G ∙ t ⊢ u / v ⟶* Y / α ∙ β) →
		    i ≋ g →
		    i ∈ rs

		  mutual
		    Complete : ∀ {v w} → EState w v → Set
		    Complete ω = Complete₀ (Latest ω) × Complete* ω

		    Complete* : ∀ {v w} → EState w v → Set
		    Complete* (start rs) = ⊤
		    Complete* (step ω rs) = Complete ω
	\end{code}
\end{frame}

\begin{frame}
	\frametitle{Related Work}

\end{frame}

\begin{frame}
	\frametitle{Conclusion}
	\begin{itemize}

		\item

			Managed to implement and show soundness and completeness for the
			parser. 3kloc Agda.

		\item

			Parser is slow, probably due to poor choice of data structures.
			Possibly also uneliminated proofs, but probably data structures.

		\item



	\end{itemize}
\end{frame}

\begin{frame}
	\frametitle{Questions?}
\end{frame}
