% CREATED BY DAVID FRISK, 2016
\chapter{Earley's Algorithm}

	The Earley parsing algorithm, named after its inventor Jay Earley, was
	invented by Jay Earley in 1881~\cite{Earley}. It is a top-down parsing
	algorithm, meaning that it starts with analysing the large-scale structure
	if its input and using that as guidance for what analysis is needed when
	diving deeper into more fine-scale structure and eventually single tokens.
	Eearley parsing is interesting in that it is able to perform well on
	several commonly used types of grammars, such as the LR(k) grammars, while
	also being able to parse all context-free grammars. This section will first
	give a high-level description of the algorithm, followed by a formalization
	of its mechanics, and finally give our implementation of the algorithm in
	Agda.

	\section{Description}
		
		The Earley algorithm works by keeping track of a set of items
		representing possible partial parses for each consumption state of the
		input sequence. It advances over the sets from left to right, with each 
		step to the right representing the consumption, or \emph{`scanning`} of 
		one token. For each set, all items in it are iterated over, and they 
		are each checed if it and any item in the current or any previous set 
		together imply that the parsing can continue even further, resulting in 
		a new item. All new items are added to the current set, and the process 
		is then repeated until no new unique items can be formed.

		The items are of the form $(X \mapsto \alpha \cdot \beta, i)$, where
		$X$ is the nonterminal this item represents a partial parse of,
		$\alpha$ is the sequence of terminals and non-terminals that have
		already been successfully parsed, $\beta$ is what remains to be parsed,
		and $i$ is the index of the set where the parsing attempt of $X$ was
		started. New items can be constructed from previous items in one of
		three ways:

		\begin{itemize}
			\item 
				If $\beta$ is of the form $Y\delta$, meaning that for this item
				to be completed, a $Y$ must now be parsed at this position, a 
				new item $(Y \mapsto \epsilon \cdot \gamma, j)$ is added to the 
				current set.
			
			\item
				If $\beta$ is of the form $\epsilon$, meaning that this item
				could accept the string of tokens satring at $i$ and ending at
				$j$, we must find out why the parsing of $X$ was attempted in
				the first place. Any item $(Y \mapsto \alpha \cdot X\delta, k)$
				in set $i$  could have predicted this item, and as such, we
				will add a new item $(Y \mapsto \alpha X \cdot \delta, k)$ for 
				each such item.
			
			\item
				If $\beta$ is of the form $a\delta$, and the next token in the 
				input sequence is $a$, a new item 
				$(X \mapsto \alpha a \cdot \delta, i)$ is added to the next 
				set.

		\end{itemize}
		
		The algorithm is initialized with all sets empty except for the first
		one, where a single item $(S_0 \mapsto \epsilon \cdot S, 0)$ is
		inserted.  $S$ is the starting symbol, that is, the non-terminal which
		we want to find out if the input sequence conforms to or not. $S_0$ is
		a special marker that is only used for the initial item. When parsing
		of the input sequence is complet, iff in the last set there is an item
		$(S_0 \mapsto S \cdot \epsilon, 0)$ then the parser accepts, otherwise
		the parser rejects the input.

		Recreating the parse tree can be done via backtracking from the item 
		sets. The nifty thing about this algorithm is that each item only can 
		exist once in each item set, if the grammar is ambiguous and items can 
		be derived is several different ways (or even an unbounded number of 
		ways), the different derivations will all share their common items, 
		which gives the Earley parsing algorithm the ability to correctly 
		analyse ambiguous grammars.

	\section{Formalization}
		
		We will begin by defining a proposition for how and what items can be 
		added to the different sets.

		\begin{table}[h]
			\centering
			\begin{tabular}{cccc}
				initial & scanner & predict & complet
			\end{tabular}
		\end{table}

		A similar proposition can be constructed in Agda:

		\begin{code}

			data _∙_⊢_/_⟶*_/_∙_ (G : CFG) :
			  (t u v : T *) -> N -> (N ∣ T) * -> (N ∣ T) * -> Set where
			
			  initial : {u : T *} {α : (N ∣ T) *} ->
			    CFG.rules G ∋ (CFG.start G , α) ->
			    G ∙ u ⊢ u / u ⟶* CFG.start G / ε ∙ α
			
			  scanner : {t u v : T *} {a : T} {X : N} {α β : (N ∣ T) *} ->
			    G ∙ t ⊢ u / a ∷ v ⟶* X / α ∙ r a ∷ β ->
			      G ∙ t ⊢ u / v ⟶* X / α ←∷ r a ∙ β
			
			  predict : {t u v : T *} {X Y : N} {α β δ : (N ∣ T) *} -> 
			    CFG.rules G ∋ (Y , δ) ->
			    G ∙ t ⊢ u / v ⟶* X / α ∙ l Y ∷ β ->
			      G ∙ t ⊢ v / v ⟶* Y / ε ∙ δ
			
			  complet : {t u v w : T *} {X Y : N} {α β γ : (N ∣ T) *} ->
			    G ∙ t ⊢ u / v ⟶* X / α ∙ l Y ∷ β ->
			    G ∙ t ⊢ v / w ⟶* Y / γ ∙ ε ->
			      G ∙ t ⊢ u / w ⟶* X / α ←∷ l Y ∙ β

		\end{code}

		Here, a value with type $G \cdot t \vdash u / v \rightarrow^* X /
		\alpha \cdot \beta$ would correspond to an item $(X \mapsto \alpha
		\cdot \beta, u)$ (using $u$ as a ramainder string rather than a
		consumed-index) in set $v$ that was generated while parsing the
		sequence $t$ with grammar $G$. 
		
		There are some differences between the propositions we first wrote for
		Earley parsers and the data type given above. The addition of an
		explicit grammar and sequence-to-be-parsed change little about the how
		the constructors can be used, but make it easier to reason about
		different grammars and ensure that the initial item will indeed only be
		put in the initial item set. The initial item has also been changed
		somewhat. Instead of introducing a special marker for the initial item,
		all rules for the starting symbol are now allowed as initial items.
		This also has little effect on the propositions, but introduces a
		useful (and probably expected) invariant: $\alpha \concat \beta \in 
		\textrm{CFG.rules } G$.
	
		We also provide proofs that these propositions are sound and complete
		with respect to the parsing propositions provided in chapter
		\ref{Parsing}:

		\begin{code}

			sound : ∀ {G t u v w X α β} ->
			  G ∙ t ⊢ u / v ⟶* X / α ∙ β ->
			    G ⊢ v & w ∈ β ->
			    G ⊢ u & w ∈ α ++ β
			sound (initial x) b = b
			sound (scanner g) b = sound g (term b)
			sound (predict x g) b = b
			sound (complet g g₁) b =
			  let x₁ = sound g₁ empt in
			  let x₂ = in-g g₁ in
			  let x₃ = conc x₂ x₁ b in
			  sound g x₃
			
		\end{code}
	
		Soundness is fairly straightforward. The \codett{in-g} function proves 
		the invariant $\alpha \concat \beta \in \textrm{CFG.rules } G$ for the 
		Earley propositions. The function as shown here is not entirely 
		complete: there are several instances of string concatenation that 
		are not normalized away, but these are not directly relevant to the 
		proof at hand. Completness follows in a similar fashion:
		
		\begin{code}

			complete : ∀ {t u v w X α β} {G : CFG} ->
			  G ∙ t ⊢ u / v ⟶* X / α ∙ β ->
			  G ⊢ v & w ∈ β ->
			    G ∙ t ⊢ u / w ⟶* X / α ++ β ∙ ε
			complete a empt = a
			complete a (conc x g g₁) =
			  let x₁ = predict x a in
			  let x₂ = complete x₁ g in
			  let x₃ = complet a x₂ in
			  complete x₃ g₁
			complete a (term g) = complete (scanner a) g
	
		\end{code}

		Like with the soundness proof some parts related to string 
		concatenation equalities have been elided. Being both sound and 
		complete, we expect the propositions we have given for Earley parsing
		to be reasonable enough to be useful for proving the correctness of an 
		implementation of the algorithm. The close similarities between the 
		constructors for the Earley porpositions and the steps in the algorithm 
		itself will also make it easier to show these properties for an 
		implementation.
	
	\section{Implementation}

		The implementation, like the proofs surrounding it, is written in cold 
		hard Agda

		\begin{code}

			record Item (w : T *) (v : T *) : Set where
			  constructor _∘_↦_∘_
			  field
			    Y : N
			    u : T *
			    α β : (N ∣ T) *
			    .{χ} : CFG.rules G ∋ (Y , α ++ β)
			    .{ψ} : (Σ λ t -> t ++ u ≡ w)        -- u is a suffix of w
	
		\end{code}
		
		Here, $Y \cdot u \mapsto \cdot \beta : \textrm{Item } w\ v$ represents
		an Earley item $(Y \mapsto \alpha \cdot \beta , u)$ that was generated
		when parsing input sequence $w$ with remainder $v$. Each item also 
		carries with it a proof that it matches some rule in the grammar 
		($\chi$), as well as a proof that the index $u$ fits somewhere in the 
		entire sequence $w$ ($\psi$). These proofs are declared irrelevant so 
		as to not 'get in the way' of proofs of equality and uniqueness later 
		on.

		We continue with defining the item sets:
	
		\begin{code}
	
			data WSet : T * -> T * -> Set where
			  start : {v : T *} ->
			    (rs : Item v v * ) ->
			    WSet v v
			
			  step : {a : T} {w v : T *} ->
			    (ω : WSet w (a ∷ v)) ->
			    (rs : Item w v * ) ->
			    WSet w v
	
		\end{code}
		
		These are essentially lists of item sets (although each item set has a 
		slightly different type due to the definition of the items), with each 
		item set being a list of items. These sets will then be constructed in 
		sequence in the same way that is indicated by the Earley porpositions.
		
		In our implementation, the predict and complete steps are run in unison
		and only when they are completed is the scanner step run on all of the 
		generated items. This is possible because the predict and complete 
		steps only depend on items in the current and (for complete) previous 
		sets, whereas the scanning step depends only on items in the current 
		set, but generates items in the next one. This means that the items 
		generated from the scanning step from the current set are irrelevant 
		for the predict and complete steps in the current set.

		The scanning step is very simple:
		
		\begin{code}
	
			scanner₀ : ∀ {w v} ->
			  (a : T) ->
			  Item w (a ∷ v)* ->
			  Item w v *
			scanner₀ a ε = ε
			scanner₀ a ((X ∘ u ↦ α ∘ ε) ∷ rs) = scanner₀ a rs
			scanner₀ a ((X ∘ u ↦ α ∘ l Y ∷ β) ∷ rs) = scanner₀ a rs
			scanner₀ a ((X ∘ u ↦ α ∘ r b ∷ β [ χ ∘ ψ ]) ∷ rs) with decidₜ a b
			... | no x = scanner₀ a rs
			... | yes refl = 
			  (X ∘ u ↦ α ←∷ r a ∘ β [ v-step χ ∘ ψ ]) ∷ (scanner₀ a rs)
			
			scanner : {w v : T *} ->
			  (a : T) ->
			  WSet w (a ∷ v) ->
			  WSet w v
			scanner a ω = step ω (scanner₀ a (Sₙ ω))
		
		\end{code}
		
		Here, \codett{scanner₀} filters through a list of items for items
		that can be scanned, and creates a list of the appropriate new itmes
		for those that can. \codett{scanner} then packages this appropriately
		in a \codett{WSet}. \codett{Sₙ} returns the outermost set from a
		\codett{WSet}.
		
		The complete and predict stages are implemented similarly, and their 
		implementations have been omitted here:
		
		\begin{code}
			
			complete : ∀ {v w} -> (i : Item w v) -> Item.β i ≡ ε ->
			  WSet w v -> Item w v *
		
			predict : ∀ {v w Y β} -> (i : Item w v) -> Item.β i ≡ l Y ∷ β ->
			  WSet w v -> Item w v *

		\end{code}
		
 		The complete and predict stages do, however work together and they are
 		therefore, unlike the scannig step, from now on merged into a single
 		action:
 		
		\begin{code}
			
			pred-comp₀ : ∀ {v w} ->
			  (i : Item w v) ->
			  (ω : WSet w v) ->
			  Σ {Item w v *} λ as -> Unique as
			pred-comp₀ i@(X ∘ u ↦ α ∘ ε) w = deduplicate (complete i refl w)
			pred-comp₀ i@(X ∘ u ↦ α ∘ r a ∷ β) w = σ ε u-ε
			pred-comp₀ i@(X ∘ u ↦ α ∘ l Y ∷ β) w = deduplicate (predict i refl w)
			
			pred-comp₁ : {w n : T *} -> (ω : WSet w n) ->
			  (ss : Item w n *) -> (rs : Item w n *) -> Item w n *
		
		\end{code}
			
		
		\begin{code}
			
			pred-comp-w₂ : {w n : T *} ->
			  (ω : WSet w n) ->
			  (ss : Item w n *) ->
			  (rs : Item w n *) ->
			  (m : ℕ) ->
			  (p : suc (length (Σ.proj₁ (all-rules {w} {n}) \\ ss)) ≤ m) ->
			  Unique (rs ++ ss) ->
			  WSet w n
			pred-comp-w₂ {n} ω ss rs zero () q
			pred-comp-w₂ {n} ω ss ε (suc m) p q = Wₙ ω ss
			pred-comp-w₂ {n} ω ss rs@(r₁ ∷ _) (suc m) p q =
			  let x₁ = pred-comp-w₁ ω ss rs in
			  let x₂ = x₁ \\ (rs ++ ss) in
			  let p₁ = wf-pcw₃ (Σ.proj₀ all-rules) p q in
			  let p₂ = wf-pcw₂ x₁ (rs ++ ss) q in
			  pred-comp-w₂ ω (rs ++ ss) x₂ m p₁ p₂

			in-all : ∀ {j} -> (i : Item j) -> all-rules j ∋ i
			
			pred-comp-w : ∀ {w v} -> WSet w v -> WSet w v
			
			step-w : ∀ {w a v} ->
			  WSet w (a ∷ v) ->
			  WSet w v
			step-w {w} {a} {v} ω = scanner-w a (pred-comp-w ω)
			
			parse₀ : ∀ {w v} ->
			   WSet w v ->
			   WSet w ε
			parse₀ {v = ε} w = pred-comp-w w
			parse₀ {v = x ∷ v} w = parse₀ (step-w w)
			
			parse : ∀ w -> WSet w ε
			parse w =
			  let x₁ = lookup (CFG.start G) (CFG.rules G) in
			  parse₀ (start (gen-initials w x₁))
		
		\end{code}
