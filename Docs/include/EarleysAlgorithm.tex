\AgdaHide{ \begin{code}

open import base

module EarleysAlgorithm (N T : Set) (ₙ : Dec N) (eqₜ : Dec T) where

open import grammar-mini N T
\end{code}}
\chapter{Earley's Algorithm} \label{Earleys}

	The Earley parsing algorithm, named after its inventor Jay Earley, was
	first described in 1968~\cite{Earley}. It is a top-down parsing algorithm,
	meaning that it starts with analyzing the large-scale structure of its
	input and uses that as guidance for what analysis is needed when diving
	deeper into more fine-scale structure and eventually single tokens.  Earley
	parsing is interesting in that it is able to perform well on several
	commonly used types of grammars, such as the grammars parseable by LR
	parsers, while also being able to parse all context-free grammars. This
	section will first give a high-level description of the algorithm, followed
	by a formalization of its mechanics, and finally describe our
	implementation of the algorithm in Agda.

	\section{Description}

		The Earley algorithm works by keeping track of a set of items
		representing possible partial parses for each position in the input
		sequence. It advances over the sets from left to right, with each step
		to the right representing the consumption, or \emph{`scanning'} of one
		token. When completing each set, all items generated by scanning from
		the previous set are first analyzed to determine if any further action
		can be taken on any of them that will generate new items. All new items
		are then added into the set (or, when doing scanning, to the next set),
		and all items in the current set are then analyzed again to determine
		if any new items can be constructed. This is repeated until there are
		no more items to create, and the process then continues to the next
		item set.

		The items are of the form $\earley{X}{\alpha}{\beta}{i}$, where $X$ is
		the non-terminal this item is trying to analyze, $\alpha$ is the
		sequence of terminals and non-terminals that have already been
		successfully parsed, $\beta$ is what remains to be parsed, and $i$ is
		the index of the set where the parsing attempt of $X$ was started. We
		will use $\mathbb{S}_i$ to mean the item set at position $i$, between
		the $i$th and the $i+1$st character. New items can be constructed from
		previous items in three ways:

		\begin{itemize}
			\item

				If $\beta$ is of the form $Y\delta$, meaning that for this item
				to be completed, a $Y$ must now be parsed at this position, a
				new item $\earley{Y}{\epsilon}{\gamma}{j}$, where $j$ is the
				current position, is added to the current set.

			\item

				If $\beta$ is of the form $\epsilon$, meaning that this item
				could accept the string of tokens staring at $i$ and ending at
				the current position, $j$, we must find out why the parsing of
				$X$ was attempted in the first place. Any item $(Y \mapsto
				\alpha \cdot X\delta, k)$ in set $i$  could have predicted this
				item, and as such, we will add a new item $\earley{Y}{\alpha
				X}{\delta}{k}$ for each such item.

			\item

				If $\beta$ is of the form $a\delta$, and the next token in the
				input sequence is $a$, a new item $\earley{X}{\alpha
				a}{\delta}{i}$ is added to the next set.

		\end{itemize}

		The algorithm is initialized with all sets empty except for the first
		one, where a single item $\earley{S_0}{\epsilon}{S}{0}$ is inserted.
		$S$ is the starting symbol, that is, the non-terminal which we want to
		find out if the input sequence conforms to or not. $S_0$ is a special
		marker that is only used for the initial item. When parsing of the
		input sequence is complete, if in the last set there is an item
		$\earley{S_0}{S}{\epsilon}{0}$ then the parser accepts, otherwise the
		parser rejects the input.

		Recreating the parse tree can be done via backtracking from the item
		sets. The nifty thing about this algorithm is that each item can only
		exist once in each item set, if the grammar is ambiguous and items can
		be derived is several different ways (or even an unbounded number of
		ways), the different derivations will all share their common items,
		which gives the Earley parsing algorithm the ability to correctly
		analyze ambiguous grammars.

		\todo{I guess an example of an Earley parser run would be helpful in
		this section. (And in your thesis defense!)}

	\section{Formalization}

		We will begin by defining a proposition for how and what items can be
		added to the different sets.

		\begin{table}[h]
			\centering
			\begin{tabular}{cc}
				\( \displaystyle \frac{}
					{\earley{S_0}{\epsilon}{S}{0} \in \mathbb{S}_0}
					~~\textrm{start } S
					\) &
				\( \displaystyle \frac
					{\earley{X}{\alpha}{Y\beta}{i} \in \mathbb{S}_j}
					{\earley{Y}{\epsilon}{\delta}{j} \in \mathbb{S}_j}
					~~Y \mapsto \delta
					\)
				\\~&~\\
				\( \displaystyle \frac
					{\earley{X}{\alpha}{a\beta}{i} \in \mathbb{S}_j}
					{\earley{X}{\alpha a}{\beta}{i} \in \mathbb{S}_{j+1}}
					\) &
				\( \displaystyle \frac
					{
						\earley{X}{\alpha}{Y\beta}{i} \in \mathbb{S}_j~~~~
						\earley{Y}{\gamma}{\epsilon}{j} \in \mathbb{S}_k
						}
					{\earley{X}{\alpha Y}{\beta}{i} \in \mathbb{S}_k}
					\)
			\end{tabular}
		\end{table}

		These inference rules match closely with the descriptions of how new
		items can be constructed, together with the initial item, and a similar
		proposition can be constructed in Agda:

		\AgdaHide{\begin{code}
			infixl 10 _∙_⊢_/_⟶*_/_∙_
		\end{code}}
		\begin{code}
			data _∙_⊢_/_⟶*_/_∙_ (G : CFG) (input : T *) :
			  (u v : T *) → N → (N ∣ T) * → (N ∣ T) * → Set where

			  initial : {α : (N ∣ T) *} →
			    (CFG.start G , α) ∈ CFG.rules G →
			    G ∙ input ⊢ input / input ⟶* CFG.start G / ε ∙ α

			  scanner : {u v : T *} {a : T} {X : N} {α β : (N ∣ T) *} →
			    G ∙ input ⊢ u / a ∷ v ⟶* X / α ∙ (r a ∷ β) →
			    G ∙ input ⊢ u / v ⟶* X / (α ←∷ r a) ∙ β

			  predict : {u v : T *} {X Y : N} {α β δ : (N ∣ T) *} →
			    CFG.rules G ∋ (Y , δ) →
			    G ∙ input ⊢ u / v ⟶* X / α ∙ (l Y ∷ β) →
			    G ∙ input ⊢ v / v ⟶* Y / ε ∙ δ

			  complet : {u v w : T *} {X Y : N} {α β γ : (N ∣ T) *} →
			    G ∙ input ⊢ u / v ⟶* X / α ∙ (l Y ∷ β) →
			    G ∙ input ⊢ v / w ⟶* Y / γ ∙ ε →
			    G ∙ input ⊢ u / w ⟶* X / (α ←∷ l Y) ∙ β
		\end{code}
		\AgdaHide{\begin{code}
			in-g : ∀ {G t u v X α β} →
			  G ∙ t ⊢ u / v ⟶* X / α ∙ β →
			    CFG.rules G ∋ (X , α ++ β)
			in-g (initial x) = x
			in-g (scanner g) = in₂ (λ t → (_ , t) ∈ _) (in₀ _ _ _) (in-g g)
			in-g (predict x g) = x
			in-g (complet g g₁) = in₂ (λ t → (_ , t) ∈ _) (in₀ _ _ _) (in-g g)

			suff-g₃ : ∀ {G t u v X α β} →
			  G ∙ t ⊢ u / v ⟶* X / α ∙ β →
			    Σ λ s → s ++ v ≡ u
			suff-g₃ (initial x) = σ ε refl
			suff-g₃ (scanner g) with suff-g₃ g
			suff-g₃ (scanner {a = a} g) | σ p₁ p₀ = σ (p₁ ←∷ a) (trans (sym (in₀ _ _ _)) (sym p₀))
			suff-g₃ (predict x g) = σ ε refl
			suff-g₃ (complet g g₁) with suff-g₃ g , suff-g₃ g₁
			suff-g₃ (complet g g₁) | σ p₁ p₀ , σ q₁ q₀ =
			  σ (p₁ ++ q₁) (trans (trans (assoc-++ p₁ _ _) (app (p₁ ++_) (sym q₀))) (sym p₀))

			suff-g₂ : ∀ {G t u v X α β} →
			  G ∙ t ⊢ u / v ⟶* X / α ∙ β →
			    Σ λ s → s ++ v ≡ t
			suff-g₂ (initial x) = σ ε refl
			suff-g₂ (predict x g) = suff-g₂ g
			suff-g₂ (complet g g₁) = suff-g₂ g₁
			suff-g₂ (scanner g) with suff-g₂ g
			suff-g₂ (scanner {a = a} g) | σ p₁ p₀ =
			  σ (p₁ ←∷ a) let x₁ = in₀ a p₁ _ in trans (sym x₁) (sym p₀)

			suff-g₁ : ∀ {G t u v X α β} →
			  G ∙ t ⊢ u / v ⟶* X / α ∙ β →
			    Σ λ s → s ++ u ≡ t
			suff-g₁ (initial x) = σ ε refl
			suff-g₁ (scanner g) = suff-g₁ g
			suff-g₁ (predict x g) = suff-g₂ g
			suff-g₁ (complet g g₁) = suff-g₁ g

		\end{code}}
		We use the notation \codett{α ←∷ r a} for appending \codett{a} to the
		end of \codett{α}.

\newcommand{\tdrop}{\mathsf{drop}}
\newcommand{\tlength}{\mathsf{length}}
\newcommand{\vinput}{\mathit{input}}
\newcommand{\rslash}{\mathrel{{}/{}}}


		A value with type $G \cdot \vinput \vdash u \rslash v \rightarrow^* X
		\rslash \alpha \cdot \beta$ can be thought of as corresponding to an item
		$\earley{X}{\alpha}{\beta}{u} \in \mathbb{S}_v$. However, the Agda
		version uses $u$ and $v$ as remainders of the input string rather than
		indexes into it.  Given an Earley item {$\earley{X}{\alpha}{\beta}{i}
		\in \mathbb{S}_j$, the corresponding type for the data type in Agda
		would then be $G \cdot \vinput \vdash
		(\tdrop\;i\;\vinput) \rslash (\tdrop\;j\;\vinput)
		\rightarrow^* X \rslash \alpha \cdot \beta$, where $\tdrop$ removes a specified number of
		elements from a list. Going the other way, $G \cdot \vinput
		\vdash u \rslash v \rightarrow^* X \rslash \alpha \cdot \beta$ would correspond to
		an Earley item $\earley{X}{\alpha}{\beta}{n -
		\tlength\;u} \in \mathbb{S}_{n -
		\tlength\;v}$}, where $\tlength$ returns the number of
		elements in a list and $n = \tlength\;\vinput$.

		There are some other differences between the propositions we first
		wrote for Earley parsers and the data type given above. The addition of
		an explicit grammar and input sequence change little about the how the
		constructors can be used, but make it easier to reason about different
		grammars and ensure that the initial item will indeed only be put in
		the initial item set. The initial item has also been changed somewhat.
		Instead of introducing a special marker for the initial item, all rules
		for the starting symbol are now allowed as initial items.  This also
		has little effect on the propositions, but introduces a useful
		invariant: $X \mapsto \alpha \concat \beta \in \textsf{rules } G$.

		We also provide proofs that these propositions are sound and complete
		with respect to the parsing propositions provided in
		Chapter~\ref{Parsing}:

		\begin{code}
			sound₁ : ∀ {G t u v w X α β} →
			  G ∙ t ⊢ u / v ⟶* X / α ∙ β →
			    G ⊢ v ∥ w ∈ β →
			    G ⊢ u ∥ w ∈ α ++ β
			sound₁ (initial x)     b = b
			sound₁ (scanner g)     b = (sound₁ g (term b))
			sound₁ (predict x g)   b = b
			sound₁ (complet g g₁)  b =
			  let x₁ = (sound₁ g₁ empt) in
			  let x₂ = (in-g g₁) in
			  let x₃ = conc x₂ x₁ b in
			  (sound₁ g x₃)
		\end{code}

		Soundness is fairly straightforward. The \codett{in-g} function proves
		the invariant $\alpha \concat \beta \in \textrm{rules } G$ for the
		Earley propositions. The function as shown here is not entirely
		complete: there are several instances of string concatenation that do
		not disappear during normalization, but these are not directly relevant
		to the proof at hand. Completeness follows in a similar fashion:

		\begin{code}
			complete₁ : ∀ {t u v w X α β} {G : CFG} →
			  G ∙ t ⊢ u / v ⟶* X / α ∙ β →
			    G ⊢ v ∥ w ∈ β →
			  G ∙ t ⊢ u / w ⟶* X / α ++ β ∙ ε
			complete₁ a empt = a
			complete₁ a (conc x g g₁) =
			  let x₁ = predict x a in
			  let x₂ = complete₁ x₁ g in
			  let x₃ = complet a x₂ in
			  let x₄ = complete₁ x₃ g₁ in
			  x₄
			complete₁ a (term g) =
			  let x₁ = complete₁ (scanner a) g in
			  x₁
		\end{code}

		Like with the soundness proof some parts related to string
		concatenation equalities have been elided. Being both sound and
		complete, we expect the propositions we have given for Earley parsing
		to be reasonable enough to be useful for proving the correctness of an
		implementation of the algorithm. The close similarities between the
		constructors for the Earley propositions and the steps in the algorithm
		itself will also make it easier to show these properties for an
		implementation.

	\section{Implementation}
		\AgdaHide{\begin{code}
			module parser (G : CFG) where
			  open import count N T ₙ eqₜ

			  v-step : ∀ {Y α x β} →
			    CFG.rules G ∋ (Y , α ++ (x ∷ β)) → CFG.rules G ∋ (Y , (α ←∷ x) ++ β)
			  v-step {Y} {α} {x} {β} v =
			    in₂ (λ x → CFG.rules G ∋ (Y , x)) (in₀ x α β) v

			  v-unstep : ∀ {Y α x β} →
			    CFG.rules G ∋ (Y , (α ←∷ x) ++ β) → CFG.rules G ∋ (Y , α ++ (x ∷ β))
			  v-unstep {Y} {α} {x} {β} v =
			    in₂ (λ x → CFG.rules G ∋ (Y , x)) (sym (in₀ x α β)) v
		\end{code}}

		We start by creating a representation for the Earley items:

		\begin{code}
			  record Item (w : T *) (v : T *) : Set where
			    constructor _∘_↦_∘_
			    field
			      Y : N
			      u : T *
			      α β : (N ∣ T) *
			      .{χ} : CFG.rules G ∋ (Y , α ++ β)
			      .{ψ} : (Σ λ t → t ++ u ≡ w)        -- u is a suffix of w
		\end{code}

		Here, $Y \cdot u \mapsto \alpha \cdot \beta : \textrm{Item } w\ v$
		represents an Earley item $\earley{Y}{\alpha}{\beta}{u}$ that was
		generated when parsing the input sequence $w$ up to, but not including
		the remainder $v$. Each item also carries with it a proof that it
		matches some rule in the grammar ($\chi$), as well as a proof that the
		index $u$ fits somewhere in the entire sequence $w$ ($\psi$). These
		proofs are declared irrelevant which means that they must be part of
		every item, but are not considered part of the data type for e.g.
		propositional equality, since we only care that the validity proofs
		exist not that equal items must have equal validity proofs (although
		they likely will). We could also have provided the invariant of $v$
		being a suffix of $u$, or $v$ of $w$, but these were not used in our
		implementation or proofs.

		\AgdaHide{\begin{code}
			  infixl 3 _∘_↦_∘_
			  pattern _∘_↦_∘_[_∘_] Y u α β χ ψ = (Y ∘ u ↦ α ∘ β) {χ} {ψ}
			  infixl 3 _∘_↦_∘_[_∘_]

			  eq-α :
			    (a b : (N ∣ T)*) →
			    a ≡ b ??
			  eq-α = eq-* (eq-∣ ₙ eqₜ)

			  eq-T* : (a b : T *) → a ≡ b ??
			  eq-T* = eq-* ₜ

			  eq-rule : (a b : N × (N ∣ T) *) → a ≡ b ??
			  eq-rule = eq-× ₙ eq-α

			  eq-item : ∀ {w v} → (a b : Item w v) → a ≡ b ??
			  eq-item (X ∘ i ↦ α ∘ β) (Y ∘ j ↦ γ ∘ δ) with eq-×₄ ₙ eq-T* eq-α eq-α (X , i , α , β) (Y , j , γ , δ)
			  eq-item (X ∘ i ↦ α ∘ β) (X ∘ i ↦ α ∘ β) | yes refl = yes refl
			  eq-item (X ∘ i ↦ α ∘ β) (Y ∘ j ↦ γ ∘ δ) | no x = no λ {refl → x refl}

			  open Unique Item eq-item
		\end{code}}

		We continue with defining the item sets:

		\begin{code}
			  data EState : T * → T * → Set where
			    start : {v : T *} →
			      (rs : Item v v * ) →
			      EState v v

			    step : {a : T} {w v : T *} →
			      (ω : EState w (a ∷ v)) →
			      (rs : Item w v * ) →
			      EState w v
		\end{code}
		\AgdaHide{\begin{code}
			  V : {w v : T *} →
			    EState w v →
			    Σ λ t → t ++ v ≡ w
			  V {w} {w} (start rs) = σ ε refl
			  V {w} {v} (step ω rs) with V ω
			  V {w} {v} (step {a} ω rs) | σ p₁ p₀ =
			    σ (p₁ ←∷ a) (trans (sym (in₀ a p₁ v)) (sym p₀))

			  -- Get the latest state.

			  Sₙ : {w v : T *} →
			    EState w v →
			    Item w v *
			  Sₙ (start rs) = rs
			  Sₙ (step w rs) = rs

			  -- Replace the latest state.

			  Wₙ : {w v : T *} →
			    (ω : EState w v) →
			    (rs : Item w v *) →
			    EState w v
			  Wₙ (start rs) rs₁ = start rs₁
			  Wₙ (step w rs) rs₁ = step w rs₁
		\end{code}}

		These are essentially lists of item sets (although each item set has a
		slightly different type due to the definition of the items), with each
		item set being a list of items. These sets will then be constructed in
		sequence in the same way that is indicated by the Earley propositions.

		In our implementation, the predict and complete steps are run together
		and only when they are completed is the scanner step run on all of the
		generated items. This is possible because the predict and complete
		steps only depend on items in the current and (for complete) previous
		sets, whereas the scanning step depends only on items in the current
		set, but generates items in the next one. This means that the items
		generated from the scanning step from the current set are irrelevant
		for the predict and complete steps in the current set.

		The scanning step is very simple:

		\begin{code}
			  scanr₀ : ∀ {w v} →
			    (a : T) →
			    Item w (a ∷ v)* →
			    Item w v *
			  scanr₀ a ε = ε
			  scanr₀ a ((X ∘ u ↦ α ∘ ε) ∷ rs) = scanr₀ a rs
			  scanr₀ a ((X ∘ u ↦ α ∘ l Y ∷ β) ∷ rs) = scanr₀ a rs
			  scanr₀ a ((X ∘ u ↦ α ∘ r b ∷ β [ χ ∘ ψ ]) ∷ rs) with ₜ a b
			  ... | yes refl = (X ∘ u ↦ α ←∷ r a ∘ β [ v-step χ ∘ ψ ]) ∷ (scanr₀ a rs)
			  ... | no x = scanr₀ a rs

			  scanr : {w v : T *} →
			    (a : T) →
			    EState w (a ∷ v) →
			    EState w v
			  scanr a ω = step ω (scanr₀ a (Sₙ ω))
		\end{code}

		Here, \codett{scanr₀} filters through a set of items for items that can
		be scanned, and creates a set of the appropriate new items for those
		that can. \codett{scanr} then packages this appropriately in a
		\codett{EState}. \codett{Sₙ} returns the outermost item set from a
		\codett{EState}. We use [brackets] to explicitly state the otherwise
		implicit proofs \codett{χ} and \codett{ψ} for items.

		The complete and predict steps are implemented similarly to the
		scanning step (although their results are not immediately packaged into
		an EState), and their implementations have been omitted here:

		\AgdaHide{\begin{code}
			  compl₀ : ∀ {u v w} →
			    (ω : EState w v) →
			    Item w u *
			  compl₀ {u} {v} w           with eq-T* u v
			  compl₀ {u} {u} w           | yes refl = Sₙ w
			  compl₀ {u} {v} (start rs)  | no x = ε
			  compl₀ {u} {v} (step w rs) | no x = compl₀ w

			  compl₁ : ∀ {u v w} →
			    (i : Item w v) → Item.β i ≡ ε →
			    Item w u * → Item w v *
			  compl₁ i@(X ∘ u ↦ α ∘ ε) refl ε = ε
			  compl₁ i@(X ∘ u ↦ α ∘ ε) refl ((Y ∘ u₁ ↦ α₁ ∘ ε) ∷ rs) = compl₁ i refl rs
			  compl₁ i@(X ∘ u ↦ α ∘ ε) refl ((Y ∘ u₁ ↦ α₁ ∘ r a ∷ β) ∷ rs) = compl₁ i refl rs
			  compl₁ i@(X ∘ u ↦ α ∘ ε) refl ((Y ∘ u₁ ↦ α₁ ∘ l Z ∷ β) ∷ rs) with ₙ X Z
			  compl₁ i@(X ∘ u ↦ α ∘ ε) refl ((Y ∘ u₁ ↦ α₁ ∘ l Z ∷ β) ∷ rs) | no x = compl₁ i refl rs
			  compl₁ i@(X ∘ u ↦ α ∘ ε) refl ((Y ∘ u₁ ↦ α₁ ∘ l X ∷ β [ χ₁ ∘ ψ₁ ]) ∷ rs) | yes refl =
			    (Y ∘ u₁ ↦ α₁ ←∷ l X ∘ β [ v-step χ₁ ∘ ψ₁ ]) ∷ compl₁ i refl rs

			  -- For a completed item X ↦ α.ε, get the set of possible ancestors (callers).

		\end{code}}
		\begin{code}
			  compl : ∀ {v w} →
			    (i : Item w v) → Item.β i ≡ ε →
			    EState w v → Item w v *
		\end{code}
		\AgdaHide{\begin{code}
			  compl i p ω = compl₁ {u = Item.u i} i p (compl₀ ω)

			  predict₀ : ∀ {v w Y β} →
			    (Σ λ t → t ++ v ≡ w) →
			    (i : Item w v) → Item.β i ≡ l Y ∷ β →
			    (Σ λ t → (t ∈ CFG.rules G) × (fst t ≡ Y)) * →
			    Item w v *
			  predict₀ ψ₁ i p ε = ε
			  predict₀ {v} ψ₁ i@(X ∘ u ↦ α ∘ l Y ∷ β) refl (σ (Y , γ) (p , refl) ∷ ps) =
			    (Y ∘ v ↦ ε ∘ γ [ p ∘ ψ₁ ]) ∷ predict₀ ψ₁ i refl ps

		\end{code}}
		\begin{code}
			  predic : ∀ {v w Y β} →
			    (i : Item w v) → Item.β i ≡ l Y ∷ β →
			    EState w v →
			    Item w v *
		\end{code}
		\AgdaHide{\begin{code}
			  predic i@(X ∘ u ↦ α ∘ l Y ∷ β) refl ω =
			    predict₀ (V ω) i refl (lookup Y (CFG.rules G))

			  deduplicate : ∀ {w v} → Item w v * → Σ λ as → Unique as
			  deduplicate ε = σ ε u-ε
			  deduplicate (x ∷ as) with elem eq-item x (Σ.proj₁ (deduplicate as))
			  deduplicate (x ∷ as) | yes x₁ = deduplicate as
			  deduplicate (x ∷ as) | no x₁ =
			    σ (x ∷ (Σ.proj₁ (deduplicate as))) (u-∷ (Σ.proj₀ (deduplicate as)) x₁)
		\end{code}}

		The complete and predict stages do, however, work together and they are
		therefore, unlike the scanning step, from now on merged into a single
		action:

		\begin{code}
			  pred-comp₀ : ∀ {v w β} →
			    (i : Item w v) →
			    (β ≡ Item.β i) →
			    (ω : EState w v) →
			    Σ {Item w v *} λ as → Unique as
			  pred-comp₀ i@(X ∘ u ↦ α ∘ ε) refl ω = deduplicate (compl i refl ω)
			  pred-comp₀ i@(X ∘ u ↦ α ∘ r a ∷ β) refl ω = σ ε u-ε
			  pred-comp₀ i@(X ∘ u ↦ α ∘ l Y ∷ β) refl ω with elem ₙ Y (nullable G)
			  pred-comp₀ i@(X ∘ u ↦ α ∘ l Y ∷ β) refl ω | no x = deduplicate (predic i refl ω)
			  pred-comp₀ i@(X ∘ u ↦ α ∘ l Y ∷ β) refl ω | yes x =
			    let i₁ = X ∘ u ↦ α ←∷ l Y ∘ β [ v-step (Item.χ i) ∘ Item.ψ i ] in
			    let x₁ = pred-comp₀ i₁ refl ω in
			    let x₂ = predic i refl ω in
			    deduplicate (i₁ ∷ (Σ.proj₁ x₁ ++ x₂))

			  pred-comp₁ : {w n : T *} → (ω : EState w n) →
			    (ss : Item w n *) → (rs : Item w n *) → Item w n *
		\end{code}
		\AgdaHide{\begin{code}
			  pred-comp₁ ω ss ε = ε
			  pred-comp₁ ω ss (r₁ ∷ rs) =
			    let x₁ = pred-comp₀ r₁ refl (Wₙ ω ss) in
			    Σ.proj₁ x₁ ++ pred-comp₁ ω ss rs
		\end{code}}

		This does not do much more than selecting the appropriate step based on
		the item to be analyzed, and mapping this over a list of items to be
		analyzed. The result is also filtered so that all generated items are
		unique. This is then put when generating all the predicted and
		completed items for an item set:

		\AgdaHide{\begin{code}
			  all-items₀ : ∀ {v w X u α β} → ∀ χ ψ →
			    (i : Item w v) → i ≡ (X ∘ u ↦ α ∘ β [ χ ∘ ψ ]) →
			    Σ {Item w v *} λ as →
			      ∀ {γ δ} → ∀ χ .ψ → (i : Item w v) → i ≡ (X ∘ u ↦ γ ∘ δ [ χ ∘ ψ ]) → (Σ λ t →
			        (t ++ δ ≡ β) × (α ++ t ≡ γ)) →
			      i ∈ as
			  all-items₀ χ ψ i@(X ∘ u ↦ α ∘ ε) refl = σ (i ∷ ε) λ
			    { χ ψ (X ∘ u ↦ γ ∘ .ε) refl (σ ε (refl , y)) → case trans (sym (++-ε α)) (sym y) of λ {refl → in-head}
			    ; χ ψ (X ∘ u ↦ γ ∘ δ) refl (σ (x₁ ∷ t) (() , y))
			    }
			  all-items₀ χ ψ i@(X ∘ u ↦ α ∘ x ∷ β) refl with all-items₀ (in₄ χ) ψ (X ∘ u ↦ α ←∷ x ∘ β [ in₄ χ ∘ ψ ]) refl
			  all-items₀ χ ψ i@(X ∘ u ↦ α ∘ x ∷ β) refl | σ p₁ p₀ = σ (i ∷ p₁) λ
			    { χ ψ (X ∘ u ↦ γ ∘ .(x ∷ β)) refl (σ ε (refl , y)) → case trans (sym (++-ε α)) (sym y) of λ {refl → in-head}
			    ; χ ψ i@(X ∘ u ↦ γ ∘ δ) refl (σ (x₁ ∷ t) (refl , y)) → in-tail (p₀ χ ψ i refl (σ t (refl , (trans (sym (in₀ _ _ _)) (sym y)))))
			    }

			  all-items₁ : ∀ {w v X u β} → ∀ χ ψ →
			    (i : Item w v) → i ≡ (X ∘ u ↦ ε ∘ β [ χ ∘ ψ ]) →
			    Σ {Item w v *} λ as →
			      ∀ {u₀ γ δ} → ∀ χ ψ → (i : Item w v) → i ≡ (X ∘ u₀ ↦ γ ∘ δ [ χ ∘ ψ ]) →
			        γ ++ δ ≡ β → (Σ λ t → t ++ u₀ ≡ u) →
			        i ∈ as
			  all-items₁ χ ψ i@(X ∘ ε ↦ ε ∘ β) refl with all-items₀ χ ψ i refl
			  all-items₁ χ ψ (X ∘ ε ↦ ε ∘ β) refl | σ p₁ p₀ = σ p₁ λ
			    { χ₁ ψ₁ (.X ∘ .ε ↦ γ ∘ δ) refl x₁ (σ ε refl) → p₀ χ₁ ψ₁ _ refl (σ γ (x₁ , refl))
			    ; χ₁ ψ₁ (.X ∘ u₀ ↦ γ ∘ δ) refl x₁ (σ (x ∷ t) ())
			    }
			  all-items₁ {w} {v} χ ψ@(σ q₀ q₁) i@(X ∘ x ∷ u ↦ ε ∘ β) refl with all-items₀ χ ψ i refl | all-items₁ {w} {v} χ (σ (q₀ ←∷ x) (trans (sym (in₀ _ _ _)) (sym q₁))) (X ∘ u ↦ ε ∘ β) refl
			  all-items₁ {w} {v} χ ψ (X ∘ x ∷ u ↦ ε ∘ β) refl | σ p₁ p₀ | σ p₂ p₃ = σ (p₁ ++ p₂) λ
			    { χ₁ ψ₁ (.X ∘ .(x ∷ u) ↦ γ ∘ δ) refl x₂ (σ ε refl) → in-l (p₀ χ₁ ψ₁ _ refl (σ γ (x₂ , refl)))
			    ; χ₁ ψ₁ (.X ∘ u₀ ↦ γ ∘ δ) refl x₂ (σ (x₁ ∷ p₄) refl) → in-r (p₃ χ₁ ψ₁ _ refl x₂ (σ p₄ refl))
			    }

			  all-items₂ : ∀ {w v} →
			    (rs : (N × (N ∣ T)*)*) → (∀ {r} → rs ∋ r → CFG.rules G ∋ r) →
			    Σ {Item w v *} λ as →
			      ∀ {X u γ δ} → ∀ χ ψ → (i : Item w v) → i ≡ (X ∘ u ↦ γ ∘ δ [ χ ∘ ψ ]) → (X , γ ++ δ) ∈ rs → i ∈ as
			  all-items₂ ε f = σ ε λ {χ₁ ψ₁ i x ()}
			  all-items₂ {w} ((Y , α) ∷ rs) f with all-items₁ (f in-head) (σ ε refl) (Y ∘ w ↦ ε ∘ α) refl
			  all-items₂ {w} ((Y , α) ∷ rs) f | σ p₁ p₀ with all-items₂ rs (f ∘ in-tail)
			  all-items₂ {w} ((Y , α) ∷ rs) f | σ p₁ p₀ | σ p₂ p₃ = σ (p₁ ++ p₂) λ
			    { χ ψ (.Y ∘ u ↦ γ ∘ δ) refl in-head → in-l (p₀ χ ψ _ refl refl ψ)
			    ; χ ψ (X ∘ u ↦ γ ∘ δ) refl (in-tail x₁) → in-r (p₃ χ ψ _ refl x₁)
			    }

			  relevant-χ : ∀ {w v} → (i : Item w v) → CFG.rules G ∋ (Item.Y i , Item.α i ++ Item.β i)
			  relevant-χ ((Y ∘ j ↦ α ∘ β) {χ}) = elem' eq-rule (Y , α ++ β) (CFG.rules G) χ

			  open ε ₜ

			  relevant-ψ : ∀ {w v} → (i : Item w v) → Σ λ t → t ++ Item.u i ≡ w
			  relevant-ψ {ε} ((Y ∘ ε ↦ α ∘ β) {χ} {ψ}) = σ ε refl
			  relevant-ψ {ε} ((Y ∘ x ∷ u ↦ α ∘ β) {χ} {p}) = void (ε₁ (Σ.proj₀ p))
			  relevant-ψ {x ∷ w} {v} (Y ∘ ε ↦ α ∘ β [ χ ∘ p ]) = σ (x ∷ w) (++-ε (x ∷ w))
			  relevant-ψ {x ∷ w} {v} (Y ∘ y ∷ u ↦ α ∘ β [ χ ∘ p ]) with ₜ x y | eq-T* w u
			  relevant-ψ {x ∷ w} {v} (Y ∘ x ∷ w ↦ α ∘ β [ χ ∘ p ]) | yes refl | yes refl = σ ε refl
			  relevant-ψ {x ∷ w} {v} (Y ∘ x ∷ u ↦ α ∘ β [ χ ∘ p ]) | yes refl | no x₁ with relevant-ψ {w} {v} (Y ∘ x ∷ u ↦ α ∘ β [ χ ∘ ε₅ (Σ.proj₀ p) x₁ ])
			  relevant-ψ {x ∷ w} {v} (Y ∘ x ∷ u ↦ α ∘ β [ χ ∘ p ]) | yes refl | no x₁ | σ q₁ q₀ = σ (x ∷ q₁) (app (x ∷_) q₀)
			  relevant-ψ {x ∷ w} {v} (Y ∘ y ∷ w ↦ α ∘ β [ χ ∘ p ]) | no x₂    | yes refl = void (ε₃ x₂ (Σ.proj₀ p))
			  relevant-ψ {x ∷ w} {v} (Y ∘ y ∷ u ↦ α ∘ β [ χ ∘ p ]) | no x₂    | no x₁ with relevant-ψ {w} {v} (Y ∘ y ∷ u ↦ α ∘ β [ χ ∘ ε₄ (Σ.proj₀ p) x₂ ])
			  relevant-ψ {x ∷ w} {v} (Y ∘ y ∷ u ↦ α ∘ β [ χ ∘ p ]) | no x₂    | no x₁ | σ q₁ q₀ = σ (x ∷ q₁) (app (x ∷_) q₀)

		\end{code}}
		\begin{code}
			  all-items : ∀ {w} {v} → Σ λ as → {i : Item w v} → i ∈ as
		\end{code}
		\AgdaHide{\begin{code}
			  all-items with all-items₂ (CFG.rules G) id
			  all-items | σ p₁ p₀ = σ p₁ λ {i} → p₀ (relevant-χ i) (relevant-ψ i) i refl (relevant-χ i)
		\end{code}}

		\begin{code}
			  pred-comp₂ : {w n : T *} →
			    (ω : EState w n) →
			    (ss : Item w n *) →
			    (rs : Item w n *) →
			    (m : ℕ) →
			    (p : suc (length (Σ.proj₁ (all-items {w} {n}) \\ ss)) ≤ m) →
			    Unique (rs ++ ss) →
			    EState w n
			  pred-comp₂ {n} ω ss rs zero () q
			  pred-comp₂ {n} ω ss ε (suc m) p q = Wₙ ω ss
			  pred-comp₂ {n} ω ss rs@(r₁ ∷ _) (suc m) p q =
			    let x₁ = pred-comp₁ ω ss rs in
			    let x₂ = x₁ \\ (rs ++ ss) in
			    let p₁ = wf-pcw₃ (Σ.proj₀ all-items) p q in
			    let p₂ = wf-pcw₂ x₁ (rs ++ ss) q in
			    pred-comp₂ ω (rs ++ ss) x₂ m p₁ p₂
		\end{code}

		\codett{pred-comp₂} takes a \codett{EState}, whose last item set is not
		assumed to contain any items, a set of Items that have already been
		predicted and completed on, and as such can be considered to be
		'inert', as no new items can be generated in the current item set based
		on these that have not already been generated, and a set of items that
		have yet to be completed and predicted on. To help convince the
		termination checker that this function is total, an upper bound on the
		number of items that can still be generated, which decreases at
		iteration, is also provided.

		At each step, new items are generated based on the items in
		\codett{rs}. The now analyzed items in \codett{rs} are added to the set
		of 'inert' items (\codett{ss}), and the newly generated items, with
		duplicates between them and the already inert items filtered out, are
		considered the new set of items to be analyzed. It is also necessary to
		prove that the computation will eventually complete, which is done by
		showing that there is an upper bound (\codett{length (Σ.proj₁
		(all-items {w} {n}) \textbackslash\textbackslash ss)}) on the number of
		items not already in the 'inert' set, and that this always decreases.
		Because \codett{rs} must contain at least one element, and the maximum
		number of items in an item set is bounded, this will always hold.

		That the maximum number of items in an item set is bounded is shown by
		constructing a function that returns all possible items in a given item
		set: \codett{all-items}. \codett{all-items} necessarily contains many
		more items than will ever be present in any actual item set in the
		parser, many of which will be un-sound. Finding the exact set of items
		for the items sets is, of course, the problem of parsing we are already
		trying to solve.

		Finally, if there are no items left in \codett{rs} (the set of items to
		be generated from), the process is complete and \codett{pred-comp₂}
		returns all generated items. Given that \codett{rs} originally
		contained all items that could be created from scanning the previous
		token from the input sequence, the item set should now be complete, and
		we can package it all up in an EState.

		\begin{code}
			  pred-comp : ∀ {w v} →
			    EState w v → EState w v
		\end{code}
		\AgdaHide{\begin{code}
			  pred-comp {v} w =
			    let x₁ = deduplicate (Sₙ w) in
			    let x₂ = (unique-++ (Σ.proj₁ x₁) ε (Σ.proj₀ x₁) u-ε λ ()) in
			    let m = suc (length (Σ.proj₁ (all-items {v}) \\ ε)) in
			    pred-comp₂ w ε (Σ.proj₁ x₁) m (≤ₛ (≤-self _)) x₂
		\end{code}}

		At last, we put this together with the scanning step, and provide a
		function for parsing an input string:

		\begin{code}
			  step₀ : ∀ {w a v} →
			    EState w (a ∷ v) →
			    EState w v
			  step₀ {w} {a} {v} ω = scanr a (pred-comp ω)

			  parse₀ : ∀ {w v} →
			     EState w v →
			     EState w ε
			  parse₀ {v = ε} w = pred-comp w
			  parse₀ {v = x ∷ v} w = parse₀ (step₀ w)

			  itemize : ∀ w →
			    Σ (λ t → (t ∈ CFG.rules G) × (fst t ≡ CFG.start G)) * →
			    Item w w *
			  itemize w ε = ε
			  itemize w (σ (X , β) p₀ ∷ rs) = 
			    (X ∘ w ↦ ε ∘ β [ fst p₀ ∘ σ ε refl ]) ∷ itemize w rs

			  parse : ∀ w → EState w ε
			  parse w =
			    let x₁ = lookup (CFG.start G) (CFG.rules G) in
			    parse₀ (start (itemize w x₁))
		\end{code}

		This concludes our implementation of the Earley parsing algorithm.

%%% Local Variables:
%%% mode: latex
%%% TeX-master: "../Report"
%%% End:
