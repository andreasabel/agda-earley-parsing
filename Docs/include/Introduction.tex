% CREATED BY DAVID FRISK, 2016
\chapter{Introduction}


%	Parser generators are commonly used for parsers of programming languages,
%	and several different generators exist today, such as Happy~\cite{Happy},
%	Bison~\cite{Bison}, and Menhir~\cite{Menhir}. Parser generators are used
%	for several reasons: Backus Naur Form (BNF) descriptions of grammars are
%	usually more concise compared to other representations, such as hand
%	written parsers. Generated parsers that are specialized for a language can
%	also be faster compared to algorithms that parse any grammar and have to
%	read and analyze the grammar of the target language during each parse.
%	Parsers are often required to be fast, and they are the first step in what
%	is usually many different passes of the entire compilation process. To this
%	end, many parser generators generate parsers based on the LR(1) or LALR(1)
%	parsing algorithms. These algorithms are designed mainly to be fast, but
%	are unfortunately not able to parse all context-free grammars. Other
%	algorithms, such as the Earley parsing algorithm~\cite{Earley}, are able to
%	parse all context-free grammars, while maintaining an O(n) time complexity
%	for many unambiguous grammars, such as all LR(k) grammars.
%
%	The BNF Converter (BNFC)~\cite{BNFC} is a parser and lexer generator
%	written in Haskell, which uses a labeled BNF description of the target
%	language to generate an abstract syntax tree (AST) representation, a
%	parser, and a pretty-printer for the target language. BNFC supports
%	multiple different back-end languages (that is, target languages for the
%	generated parser itself), but none of these, nor other parts of BNFC have
%	been formally verified. As with compilers, errors in parser generators can
%	change the semantics of programs, yielding incorrect results.

	Agda is a functional programming language with a strong type system that
	supports dependent types. Dependent types allow expressing strong
	invariants and properties of data structures and programs, and even
	constructive proofs using the type system. These proofs enable programmers
	to formally verify the correctness of programs and libraries. While the
	theory underlying Agda's type system has been shown to be
	sound~\cite{martin84}, neither its parser, nor later stages of its compiler
	have been formally verified. Errors in the implementations of these might
	change the semantics of compiled programs and proofs, with severe
	consequences.

	Verified compilers have received significant attention in the research
	community. CompCert~\cite{Leroy} is a formally verified compiler for a
	subset of the C programming language. It is written in Coq, a proof
	assistant similar to Agda, and is accompanied by Coq-proofs that it
	preserves the semantics of compiled programs.

	Parsing is the problem of analyzing the structure of a sequence of
	symbols, typically a string of characters, to determine if, and how, the
	sequence conforms to a grammar---a description of acceptable structures.
	Parsers are commonly used as parts of compilers to extract the structure of
	source code as the first step of the compilation process, but are also
	sometimes used to extract other types of structured data or to analyze
	parts of natural language.

	One class of grammars that are often used for descriptions of programming
	languages are the \emph{context-free grammars}, for which several different
	parsing algorithms exist, such as the Cocke-Younger-Kasami (CYK) parsing
	algorithm, or Valiant's algorithm.

	Others, such as \emph{LR},\footnote{LR stands for \textbf{l}eft-to-right
	scanning, \textbf{r}ightmost derivation.} \emph{Simple LR} (SLR), or
	\emph{Look-ahead LR} (LALR) parsers, are designed to parse the input
	sequence quickly. They are both asymptotically faster, having a linear time
	complexity compared to the $O(n^3)$ complexity displayed by many others,
	and practically being optimized to run efficiently on normal systems.
	These parsers do, however, restrict the types of grammars that they are
	able to parse to a subset of the context-free grammars.

	Earley's parsing algorithm~\cite{Earley} (henceforth called \emph{Earley})
	is a parsing algorithm capable of parsing all context-free grammars. Unlike
	the CYK algorithm, which has a best-case time complexity of $O(n^3)$,
	Earley has a time complexity of $O(n^3)$ only in the worst case, when
	parsing highly ambiguous grammars, while having a linear complexity for
	many grammars, including all grammars that LR, SLR, or LALR parsers accept.
	This makes the Earley parsing algorithm interesting to analyze, being able
	to parse more grammars than the LR family of parsers, while being faster
	than algorithm such as CYK or Valiant's.

	In this thesis, we formalize and implement an Earley parser in Agda, and
	verify the correctness of said parser. Additionally, we give readers an
	introduction of Agda in Chapter~\ref{Agda}, as well as a more thorough
	explanation of parsing and context-free grammars in Chapter~\ref{Parsing}.
	In Chapter~\ref{Earleys} the formalization and implementation of the Earley
	parsing algorithm are described, and their correctness is proven in
	Chapter~\ref{Correctness}. We discuss related work in
	Chapter~\ref{Related}, and finally discuss our implementation and proofs in
	Chapter\ref{Conclusion}.

%%% Local Variables:
%%% mode: latex
%%% TeX-master: "../Report"
%%% End:
