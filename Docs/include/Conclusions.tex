% CREATED BY DAVID FRISK, 2016
\chapter{Concluding Remarks}\label{Conclusion}

	In this thesis, we formalize the mechanics of an Earley parser, and we show
	them to be equivalent to several other formalizations of parsing. We
	discuss several different variants of parsing formalizations, and some
	potential impacts on their use combined with Agda's type checker.

	We also provide an implementation of the Earley parser in Agda, and prove
	its soundness and completeness with respect to the formalized mechanics,
	and therefore also with respect to the parsing formalizations. The
	soundness proof is fairly straightforward, and follows the structure of the
	algorithm implementation closely, while the completeness proof is somewhat
	more elaborate, especially when showing that all items derivable from the
	\codett{complete} step of the parsing are indeed derived. This is caused by
	a slight divergence between how the implementation handles potentially
	empty rules and how they are handled in the formalizations. One conclusion
	from this is the importance of how one defines data types used as
	propositions in Agda: a re-formulation of the definition of properties can
	significantly change the complexity of proofs, even when the equivalence of
	the new definition and the previous one is relatively easy to prove.

	Finally, we discuss the format of the output of the implemented parser, and
	argue that the completeness proof, together with the equivalence of the
	Earley formalization and one of the other parsing formalizations, is strong
	enough to show that the parser handles all ambiguities correctly, in that
	all possible parse trees are efficiently extractable from its output.

	For future work in this area, it might be interesting to look at
	generalizations of parsing algorithms similar to \cite{sikkel97}, and
	verify implementations that can make use of several different strategies
	for parsing. Unfortunately, our implementation, and others' attempting
	similar goals, are significantly less efficient compared to commonly used
	unverified parsers and parser generators (such as \cite{Menhir, Happy,
	Bison}). It would be interesting to see work on verification of
	implementations that equally performant.
