
\chapter{Related Work} \label{Related}

	Research in formal verification and parsing has been done previously by
	several groups. Below we discuss some research related to this project.

	Jourdan and Pottier~\cite{Jourdan} demonstrate a method for verifying the
	correctness of generated LR parsers. They used the fact that LR parsers are
	stack automata guided by a finite-state automaton, and designed a verifier
	that can check whether a given automaton conforms to a specified grammar.
	As such, they did not prove the correctness of the parser generator itself.
	Further, their method does not verify parsers capable of parsing any
	context-free grammar.

	Firsov and Uustalu~\cite{Firsov14} verify a CYK parsing algorithm using
	Agda. The CYK parsing algorithm only works for grammars in Chomsky normal
	form, which is not the natural representation of the grammars for most
	target languages. This means that the grammar will first have to be
	normalized, a process which also should be verified in order to have a
	fully verified parsing process. Fortunately, Firsov and Uustalu also verify
	a normalization algorithm for context-free grammars.~\cite{Firsov15} The
	normalization process can, however, increase the size of the grammar, up to
	a quadratic increase.

	Ridge~\cite{ridge11} verifies a parser generator for context-free grammars
	that is both sound and (mostly) complete. However, while it is able to
	parse all context-free grammars, it is not able to produce all syntax tree
	derivations for some ambiguous grammars. Further, generated parsers have a
	time complexity of O($n^5$), which is significantly worse than the O($n^3$)
	time complexity of, e.g., CYK. 

	Danielsson~\cite{nad2009} describes a parser combinator library written in
	Agda which guarantees that the parsing always terminates. Parser
	combinators can be a powerful way to describe a parser in a host language,
	but many existing parser combinator libraries are unable to support
	left-recursion while still terminating. Danielsson's library does not
	perform any precomputation or optimization of the parsers, but this has
	been done for other parser combinator libraries.

	Sikkel and Nijholt~\cite{sikkel97} introduce a concept of parsing schemata,
	which describe parsers at a level between grammars and parsing algorithms,
	in order to more easily reason about similarities and differences between
	parsing algorithms. The parsing schemata map closely to chart parsers, and
	Sikkel and Nijholt describe how variants of, among others, the CYK and
	Earley parsing algorithm are represented. The authors also discuss how
	modifications and optimizations to parsing algorithms can be described and
	compared between different parsing algorithms.

	Barthwal and Norrish~\cite{barthwal09} verify the correctness of SLR
	parsers generated by a parser generator. SLR (for Simple LR) parsers are
	designed mainly to reduce the size of the parse tables from LR parsers, and
	are not able to parse all grammar parseable by LR parsers. Barthwal and
	Norrish verify both soundness and completeness for all generated parsers,
	and also show that only unambiguous grammars are accepted by the parser
	generator.

	Barthwal and Norrish~\cite{barthwal13} also verify properties of pushdown
	automata for parsing using the HOL4 proof assistant. Pushdown automata form
	the basis of several parsing algorithms, and Barthwal and Norrish formalise
	several proofs relating to pushdown automata and parsing of context-free
	grammars. As such, they do not implement and verify a full parser or parser
	generator, but instead provide a formal basis of the surrounding theory on
	which other works can be built.
