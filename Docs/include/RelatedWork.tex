
\chapter{Related Work}

	Research in formal verification and parsing has been done previously by
	several groups. Below we discuss some research related to this project.

	Jourdan and Pottier~\cite{Jourdan} demonstrate a method for verifying the
	correctness of generated LR parsers. They used the fact that LR parsers are
	stack automata guided by a finite-state automaton, and designed a verifier
	that can check whether a given automaton conforms to a specified grammar.
	As such, they did not prove the correctness of the parser generator itself.
	Further, their method does not verify parsers capable of parsing any
	context-free grammar.

	Firsov and Uustalu~\cite{Firsov14} verify a CYK parsing algorithm using
	Agda. The CYK parsing algorithm only works for grammars in Chomsky normal
	form, which is not the natural representation of the grammars for most
	target languages. This means that the grammar will first have to be
	normalized, a process which also should be verified in order to have a
	fully verified parsing process. Fortunately, Firsov and Uustalu also verify
	a normalization algorithm for context-free grammars.~\cite{Firsov15} The
	normalization process can, however, increase the size of the grammar, up to
	a quadratic increase. This might be undesirable for some situations where
	very large grammars are used.

	Ridge~\cite{ridge11} verifies a parser generator for context-free grammars
	that is both sound and (mostly) complete. However, while it is able to
	parse all context-free grammars, it is not able to produce all syntax tree
	derivations for some ambiguous grammars. Further, generated parsers have a
	time complexity of O($n^5$), which is significantly worse than the O($n^3$)
	time complexity of, e.g., CYK. This can be undesirable for some
	applications.

	While these works are closely related, this thesis will aim to verify a
	parser generator for all context-free grammars. This means that a parser
	will be created for each target grammar, and ideally that this parser could
	be emitted in several different target languages. The thesis will also use
	the Earley parsing algorithm, which has not previously been formally
	verified.
