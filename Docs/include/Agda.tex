% CREATED BY DAVID FRISK, 2016

\chapter{Agda} \label{Agda}

	Agda is a dependently typed functional programming language based on
	Martin-Löf type theory \cite{?}. Dependently typed language allow creating
	types that are parameterized over some value, which makes it possible to
	create arbitrarily specific types. Types in Martin-Löf type theory can also
	be used to represent propositions in constructive logic, with values of
	those types being proofs of the proposition. Agda, and other similar
	languages, can therefore be used as proof assistants, where type checking
	in the compiler is equivalent to mechanically verifying that the proof
	encoded in the program is correct.

	In this chapter we will give a quick introduction to programs and proofs in
	Agda. It is not necessary to be experienced with programming with dependent
	types, but some knowledge of functional programming is useful.

	\section{Baby steps}

		We will start right off with some code in Agda:

		\begin{code}
			data ℕ : Set where
			  zero : ℕ
			  suc : ℕ -> ℕ
		\end{code}

		The keyword \codett{data} marks the start of the definition of a new
		data type \codett{ℕ}. It is perfectly valid to use most of Unicode for
		identifiers in Agda. \codett{ℕ} has the type \codett{Set}, which means
		\codett{ℕ} is a normal type\footnote{\codett{Set} is the name of the
		type of types, which in turn has type \codett{Set₁}, which has the type
		\codett{Set₂}, and so on.}. This data type has two constructors:
		\codett{zero} representing zero from the natural numbers, and
		\codett{suc} the successor function.  Let's take a look at a function
		definition using this data type:

		\begin{code}
			_+_ : ℕ -> ℕ -> ℕ
			zero + b = b
			suc a + b = suc (a + b)
		\end{code}

		In Agda arbitrary mix-fix operators can be introduced using \codett{\_}
		to denote the arguments of the mix-fix operator. Here, our operator
		\codett{\_+\_} takes two natural numbers (one on each side), and
		returns a natural number. Next, we pattern-match on the first argument:
		if it is zero, the result of the addition should be equal to the second
		number. If it is the successor of another number \codett{a}, the sum is
		the successor of the sum of \codett{a} and the second argument.

		\begin{code}
			data _* (T : Set) : Set where
			  ε : T *
			  _∷_ : T -> T * -> T *
		\end{code}

		The type \codett{\_*} defines lists of some type \codett{T}. The type
		argument (\codett{T : Set}) to \codett{\_*} is placed before \codett{:}
		to indicate that all constructors must use the same type argument (that
		is, the constructors do not constrain this argument in any way), but we
		could also have written \codett{data~\_*~:~Set~->Set~where}. Let's look
		at a function making use of this type:

		\begin{code}
			_++_ : {T : Set} -> T * -> T * -> T *
			ε ++ b = b
			(x ∷ a) ++ b = x ∷ (a ++ b)
		\end{code}

		The function \codett{\_++\_} takes three arguments: a type \codett{T},
		and two lists of type \codett{T}. It then returns a list of type
		\codett{T}. The first argument is surrounded by curly brackets, which
		means it is an implicit argument. When calling this function, it will
		not be necessary to provide the type as an argument. Instead, the
		compiler will try to figure out what it's value should be based on the
		other arguments. Should the compiler be unable to figure this out, we
		can provide implicit argument using curly brackets: \codett{\_++\_ {T =
		ℕ} ε ε}. The definition of the function pattern-matches on the
		constructors for \codett{T *}, and the result is the concatenation of
		the arguments.

%	\section{Dependent data types}
%

	\section{Types as propositions}

		Agda's dependent types also allow us to define more interesting data
		types:

		\begin{code}
			data _≤_ : ℕ -> ℕ -> Set where
			  ≤₀ : {n : ℕ} -> zero ≤ n
			  ≤ₛ : {m n : ℕ} -> m ≤ n -> suc m ≤ suc n
		\end{code}

		Here, like \codett{\_*}, the data type \codett{\_≤\_} takes arguments,
		but instead of taking a type as its argument, \codett{\_≤\_} takes two
		natural numbers. The first constructor, \codett{≤₀}, takes one
		(implicit) natural number \codett{n}, and returns a value of the type
		\codett{zero ≤ n}, which means that the first argument to the type must
		always be \codett{zero} when \codett{≤₀} is used. \codett{≤ₛ} takes two
		numbers (\codett{m n}), and a value of type \codett{m ≤ n}, and applies
		the successor function to the type arguments.

		Interestingly, because types of the constructors are not fully general,
		they constrain the constructible members of the type. For example, it
		is not possible to construct a value of type \codett{suc zero ≤ zero}:
		\codett{≤₀} can not be used as the first type arguments don't match
		(\codett{≤₀} can only return a value where the first type argument is
		\codett{zero}, but here we would need suc zero), and neither can
		\codett{≤ₛ}, because the second type arguments don't match (no matter
		the arguments, the second type argument of the result will be
		\codett{suc} \emph{something}, which is not equal to zero). In fact, as
		the name of the data type suggests, it is equivalent to the proposition
		\emph{less than or equal to}, as no value with type \codett{a ≤ b} can
		be constructed where \codett{a} is greater than \codett{b}.

	\section{Equality and normalization}

		Another interesting data type is propositional equality:

		\begin{code}
			data _≡_ {T : Set} (t : T) : T -> Set where
			  refl : t ≡ t
		\end{code}

		The type \codett{\_≡\_} takes two arguments, a type \codett{T}, and two
		values of that type (\codett{t} is the first, and the second is
		unnamed).  Since the constructor can only construct values which have
		the same value as both type arguments, \codett{\_≡\_} will only be
		constructible for equal values. This might feel somewhat like cheating:
		if this data type represents equality, but the arguments must be equal
		for it to be constructed, how is this not a circular definition? This
		works because of a simple judgmental equality built-in to the
		compiler: if two terms can be normalized to exactly the same
		representation, the compiler will consider them the same, and allow us
		to construct propositional equality values with them as type arguments,
		even though the exact terms might differ. Let's take a quick look at
		how normalization works:

		\begin{code}
			assoc-++ : ∀ {T} (as bs cs : T *) -> (as ++ bs) ++ cs ≡ as ++ (bs ++ cs)
			assoc-++ ε bs cs = refl
			assoc-++ (x ∷ as') bs cs = app (x ∷_) (assoc-++ as' bs cs)
		\end{code}

	 	Here, we try to construct a proof of associativity for the
	 	\codett{\_++\_} function. Neither of the two values given to
	 	\codett{\_≡\_} normalize to anything else, because we do not know what
	 	the lists \codett{as bs} and \codett{cs} look like, so we cannot know
	 	what their concatenation is, besides it being their concatenation.
	 	However when we pattern-match on \codett{as} this changes. In the first
	 	case, with \codett{as}$=\epsilon$, when normalizing the term \codett{as
	 	++ (bs ++ cs)}, the compiler will analyze the definition of
	 	\codett{\_++\_}. \codett{\_++\_} does pattern matching on \codett{as},
	 	but since we already know the value of as, the compiler normalizes this
	 	application of \codett{\_++\_} to just \codett{bs ++ cs}, based on the
	 	definition of \codett{\_++\_}, and the same happens for the left-hand
	 	side of \codett{\_≡\_}, making both sides identical.  Similarly, for
	 	the case when \codett{as} = \codett{x ∷ as'}, the type of the function
	 	normalizes to \codett{x ∷ ((as' ++ bs) ++ cs) ≡ x ∷ (as' ++ (bs ++
	 	cs))}, which we can prove by recursion (replacing \codett{as} with
	 	\codett{as'}), followed by applying \codett{(x ∷\_)} to both sides.
