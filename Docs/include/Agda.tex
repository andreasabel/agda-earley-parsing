% CREATED BY DAVID FRISK, 2016
\chapter{Agda}

	Agda is a functional programming language based on Martin-Lööf type theory
	\cite{?}. Its dependent types allow expressing constructive proofs in the
	language and ? allow creating proofs about functions in the language
	itself. Here, we will give a quick introduction to programs and proofs in
	Agda. It is not necessary to be experienced with dependent types, but some
	knowledge of functional programming is useful.
	
	\begin{code}
	
		data Nat : Set where
			zero : Nat
			suc : Nat -> Nat
	
	\end{code}

	The keyword \codett{data} here marks the start of the definition of a new
	data type \codett{Nat}. It is perfectly valid to use most of unicode for
	identifiers in Agda. \codett{Nat} has the type \codett{Set}, which is the
	type of small types. This data type has two constructors, \codett{zero}
	representing zero from the natural numbers, and \codett{suc} the sucsessor
	furction. Let's take a look at a function definition using this data type:
	
	\begin{code}
	
		_+_ : Nat -> Nat -> Nat
		zero  + b = b
		suc a + b = suc (a + b)

	\end{code}
	
	In Agda arbitrary mixfix operators can be introduced using \codett{\_} to
	denote the arguments of the mixfix operator. Here, our operator
	\codett{\_+\_} takes two natural numbers and returns a natural number
	(spoiler: we're defining addition over the natural numbers). Next, we
	pattern-match on the first argument: if it is zero, the result of the
	addition should be equal to the second number. If it is thee sucsessor of
	another number \codett{a}, the sum is th e sucsessor of the sum of a and
	the second argument.
	

	Propositions as types.
	
	\begin{code}
		
		data _<=_ : Nat -> Nat -> Set where
		    <=0 : forall n -> zero <= n
		    <=s : forall a b -> a <= b -> suc a <= suc b

	\end{code}

	\section{The Curry-Howard Sauce}
