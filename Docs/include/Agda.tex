% CREATED BY DAVID FRISK, 2016
\chapter{Agda}

	Agda is a functional programming language based on Martin-Lööf type theory
	\cite{?}. Its dependent types allow expressing constructive proofs in the
	language and ? allow creating proofs about functions in the language
	itself. Here, we will give a quick introduction to programs and proofs in
	Agda. It is not necessary to be experienced with dependent types, but some
	knowledge of functional programming is useful.

	\begin{code}
		data ℕ : Set where
		  zero : ℕ
		  suc : ℕ -> ℕ
	\end{code}

	The keyword \codett{data} here marks the start of the definition of a new
	data type \codett{Nat}. It is perfectly valid to use most of Unicode for
	identifiers in Agda. \codett{Nat} has the type \codett{Set}, which is the
	type of small types. This data type has two constructors, \codett{zero}
	representing zero from the natural numbers, and \codett{suc} the successor
	function. Let's take a look at a function definition using this data type:

	\begin{code}
		_+_ : (a b : ℕ) -> ℕ
		zero + b = b
		suc a + b = suc (a + b)
	\end{code}

	In Agda arbitrary mix-fix operators can be introduced using \codett{\_} to
	denote the arguments of the mix-fix operator. Here, our operator
	\codett{\_+\_} takes two natural numbers and returns a natural number
	(spoiler: we're defining addition over the natural numbers). Next, we
	pattern-match on the first argument: if it is zero, the result of the
	addition should be equal to the second number. If it is the successor of
	another number \codett{a}, the sum is the successor of the sum of a and
	the second argument.

	Agda's dependent types also allow us to define more interesting data types:

	\begin{code}
		data _≤_ : ℕ -> ℕ -> Set where
		  ≤₀ : {n : ℕ} -> zero ≤ n
		  ≤ₛ : {m n : ℕ} -> m ≤ n -> suc m ≤ suc n
	\end{code}

	Here, the data type \codett{\_≤\_} is parameterized on two natural numbers,
	with constructors \codett{≤₀} and \codett{≤ₛ}. Interestingly, the types of
	the constructors are not fully general, and as such constrain the
	constructible members of the type. For example, it is not possible to
	construct a value of type \codett{suc zero ≤ zero}: \codett{≤₀} can not be
	used as the first type arguments don't match, and neither can \codett{≤ₛ},
	because the second type arguments don't match. In fact, as the name of the
	data type suggests, it is equivalent to the proposition \emph{less than or
	equal to}, as no value with type \codett{a ≤ b} can be constructed where
	\codett{a} is greater than \codett{b}.

	Finally, let's take a look at these definitions:

	\begin{code}
		data _* (T : Set) : Set where
		  ε : T *
		  _∷_ : T -> T * -> T *

		data _≡_ {T : Set} (t : T) : T -> Set where
		  refl : t ≡ t

		_++_ : {T : Set} -> T * -> T * -> T *
		ε ++ b = b
		(x ∷ a) ++ b = x ∷ (a ++ b)

		assoc-++ : ∀ {T} -> (as bs cs : T *) -> (as ++ bs) ++ cs ≡ as ++ (bs ++ cs)
		assoc-++ ε bs cs = refl
		assoc-++ (x ∷ as) bs cs = app (x ∷_) (assoc-++ as bs cs)
	\end{code}

	The type \codett{\_*} defines simple lists over a data type. The type
	argument to \codett{\_*} is placed before \codett{:} to indicate that all
	constructors must use the same type argument (that is, the constructors do
	not constrain this argument in any way). \codett{\_≡\_} defines
	propositional equality and should be fairly straightforward to understand,
	and \codett{\_++\_} concatenates two lists. Finally we have a proof of
	associativity for the \codett{\_++\_} operator. Since \codett{\_≡\_} takes
	two values as arguments, it is perfectly valid to use \codett{\_++\_} as
	part of the type definition.

	\section{The Curry-Howard Isomorphism}
